% Section: Security

By default, the environment is assumed to be completely safe, so no security
measures are in place. However the \textan{} supports encrypted communication
and even client authentication.

For detailed description of individual properties, see
\ref{sssec:WebServerSettings} for server configuration and
\ref{sssec:BasicConf} for client configuration.

\subsection{SSL}

Enabling secure communication consists of setting several properties, both
on server and client side.

For server the following setting is required. Most importantly \emph{server.ssl} must
be set to \emph{true}. Then these properties will be used by the server:
\emph{server.ssl.keyStore.path} must point to the key store containing
certificate and private key for server to use. Its type is determined by
the content of property \emph{server.ssl.keyStore.type}, default value is JKS.
Property \emph{server.ssl.keyStore.password} specifies password to the key
store and property \emph{server.ssl.keyManager.password} specifies the password
for the private key. Port used for secure communication can beconfigured by
\emph{server.ssl.port}.

For client the following setting is required. Most importantly property \emph{ssl} must
be set to \emph{true}. Then these properties will be used by the server:
\emph{ssl.trustStore} must point to the trust store containing the certificate
of the server. Its type is determined by the content of property
\emph{ssl.trustStore.type}. Property \emph{ssl.trustStore.password} specifies
password to the trust store.

\subsection{Client Authentication}

For limiting the access to the webservices, client authentication can be turned
on. Only client that provides certificate from authority trusted by the server
will be served.

For server the following setting must be accomplished. Most importantly
\emph{server.ssl.clientAuth} must be set to \emph{true}. Then these properties
will be used by the server: \emph{server.ssl.trustStore.path} must point to the
trust store containing certificate of the trusted certificate authority for
server. Its type is determined by the content of property
\emph{server.ssl.trustStore.type}, default value is JKS. Property
\emph{server.ssl.trustStore.password} specifies the password to the trust
store.

For client the following setting must be accomplished. Most importantly property
\emph{ssl.clientAuth} must be set to \emph{true}. Then these properties will be
used by the server: \emph{ssl.keyStore} must point to the key store containing
the private key and certificate signed by the certificate authority trusted by
the server. Its type is determined by the content of property
\emph{ssl.keyStore.type}. Property \emph{ssl.keyStore.password} specifies password
to the key store.
