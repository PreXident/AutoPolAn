% Section: Server

This section contains all information to successfully install, configure and run
\textan{} server side application.
\comment[anyone]{Petr}{add short introduction for this section, remind configuration
options, complexity atc.}

\subsection{Installation Guide}

This section describes how to install and configure \textan{} server on a server
machine.

\subsubsection{Prerequisites}
\label{sssec:SerInstPre}

The \textan{} server requires installation of \emph{Java 8 JRE\footnote{We
recommend to use JRE from \emph{Oracle}, available at:
\url{http://www.oracle.com/technetwork/java/javase/downloads/index.html}.
Note that the application expects 64-bit JRE on 64-bit systems.}}
or a later version and a relational database system (e.g. \emph{MySQL}).
The server should be platform independent, so it runs on any system where JRE
is available, but it depends on native and 3rd party libraries. Supported operating
systems in the distribution of the \textan{} server are Linux (32 bit and 64 bit)
and Windows (32 bit and 64 bit). \comment{Petr}{Add note about processor architecture
or linux binaries are independent on it?} If you need to run \textan{} server on
different operating system, see Section \ref{sssec:unsupport}.

\subsubsection{Installation}

The \textan{} server distribution archive contains all files needed for \textan{} server to operate,
such as server binaries, native libraries, run scripts and additional resources.
For installation it is sufficient to unpack its content into any directory.

\subsubsection{Database Configuration}
\comment[Venca]{Petr}{decribe how to configure database (relational db, DBMS
independent, mysql recomendation... etc. (check Section
\ref{sssec:SerInstPre}))}

There is a sql script \emph{(create.sql)} to setup the database located in folder
\emph{Database}. It should be executed to setup the database schema before the server
starts. However this script is \emph{MySQL} specific \comment[Venca]{Venca}{check}
 and if you have other database you should replace \emph{MySQL} specific commands.

\comment[Venca]{Petr}{Change of  the underlying database is not so easy, you need
to add db driver into classpath! Is it somewhere in the documentation? }

\subsubsection{Basic Configuration}

Before the first start, it is necessary to set up the server and the database.

\comment[Petr,Jakub,Venca]{Petr}{write how to setup server}

\comment[Petr]{Petr}{add some link to \ref{sec:ServerSettings}}


\subsubsection{Starting the Server}

The server can be started by starting scripts in its root directory. Scripts for
different operating systems unfortunately do not provide the same functionality.

The start script for the Linux-based operating system (\emph{run.sh}) runs the
server application as daemon and can be used to run the server as system service.

The start script for the Microsoft Windows OS is less powerful, only runs the
server application. To run the server as a background process, the Windows Service
is required. Unfortunately, there are no components shipped with \textan{}
to make it a formal Windows Service. However, we recommend the use of \emph{Apache
ProcRun's Daemon \footnote{More information can be found on
\url{https://commons.apache.org/proper/commons-daemon/procrun.html} and in JavaDoc
documentation for \emph{cz.cuni.mff.ufal.textan.server.AppEntry} class.}}.

Note that a startup of the server may take a very long time, depends on the amount
of data, because during a initialization are created full text indexes and models
for the named entity recognizer and object assigner and the internal web server
is started after the initialization phase.

\subsubsection{Uninstallation}
The server application use only files and resources inside install directory by
default, only exception are temporary files, which are placed in system temp folder.
To uninstall the server just delete installation directory and files from your
configuration.

\subsubsection{Unsupported Systems}
\label{sssec:unsupport}
It is possible to run \textan{} server on other systems than supported ones. Only
limitations are Java 8 and 3th-party tools and libraries, specifically NameTag%
\footnote{\url{http://ufal.mff.cuni.cz/nametag}} and MorphoDiTa%
\footnote{\url{http://ufal.mff.cuni.cz/morphodita}}.

To configure the server application for specific system follow these steps:
\begin{itemize}
\item Download MorphoDiTa sources and compile its Java bindings. Copy generated
library into \emph{lib} folder in installation directory.
\item Download NameTag sources and:
	\begin{itemize}
	\item compile its Java bindings and copy generated library into \emph{lib}
	folder in installation directory.
	\item compile NameTag and copy \emph{train\_ner} into \emph{bin} folder in
	installation directory.
	\end{itemize}
\end{itemize}

\subsection{Settings}
\label{sec:ServerSettings}

\subsubsection{Web Server Settings}
\label{sssec:WebServerSettings}
This section contains a description of properties that can be used to configure
the internal web server. These properties can be set in the file \emph{server.properties}
in installation directory.

%connector
\paragraph{server.connector.host}
The particular interface to listen on. If not set or 0.0.0.0, the web server
listens on port on all interfaces.

\paragraph{server.connector.port}
The port to listen on. If not set, the web server listens on port 9500.

%thread pool
\paragraph{server.threadPool.maxThreads}
The maximum number of threads in web server thread pool. It determines a maximum
number of simultaneously opened connections. The default value is 200.

\paragraph{server.threadPool.minThreads}
The minimum number of threads in web server thread pool. The default value is 8.

\paragraph{server.threadPool.idleTimeout}
The time in milliseconds that the connection can be idle before it is closed.

%ssl
\paragraph{server.ssl}
Enable usage of SSL. When SSL is enabled, following properties must be set:
server.ssl.keyStore.path, server.ssl.keyStore.password, server.ssl.keyManager.password.
The default value is false.

\paragraph{server.ssl.keyStore.path}
The file or URL of the SSL Key store.

\paragraph{server.ssl.keyStore.password}
The password for the key store.

\paragraph{server.ssl.keyManager.password}
The password (if any) for the specific key within the key store.

\paragraph{server.ssl.keyStore.type}
The type of the key store. The default value is "JKS".

\paragraph{server.ssl.port}
The port to listen on when SSL communication is enabled. Note that the port must
be different from server.connector.port, because HTTP connection is still possible,
but it is immediately redirected to the SSL port.

%ssl client auth
\paragraph{server.ssl.clientAuth}
Enables usage of client authentication. When enabled, following properties must
be set: \emph{server.ssl.keyStore.path}, \emph{server.ssl.keyStore.password},
\emph{server.ssl.keyManager.password}. Default value is \emph{false}.

\paragraph{server.ssl.trustStore.path}
Path to the trust store file.

\paragraph{server.ssl.trustStore.password}
Password to the trust store.

\paragraph{server.ssl.trustStore.type}
The type of the trust store.

\vspace{0.75cm}
For more information about security, please consult Section \ref{sec:Security}.

\subsubsection{Database Settings}
\label{sssec:DataSettings}

\comment{Petr}{This settings are related to Persistence layer - it means also indexing}

Because of the server can connect to an arbitrary database, it is required to
specify what type of database is used. In \textan{} project is used \emph{Hibernate}
for object-relation mapping \emph{(ORM)}, \emph{JDBC} for relation database access
and \emph{c3p0} for connection pooling (see Section
\ref{DEV-sec:UsedTechnologies} in the developer documentation).

Properties to these three libraries can be set in the file \emph{data.properties}
in installation directory. The meanings of all important properties are following:

%jdbc
\paragraph{jdbc.driverClassName}
Name of the driver class made usually by database developers which enables to
interact with database.

\paragraph{jdbc.url}
 The url that points to our database. The most common url format is like this:
jdbc:[database type]:[hostname]:[port number]/[database name]
and is specific to the driver we use.

\paragraph{jdbc.user}
Username to log into the database.
\paragraph{jdbc.pass}
Password to authorize the user in database.

%c3p0 (http://www.mchange.com/projects/c3p0/#configuration_properties)
\paragraph{c3p0.maxPoolSize}
Maximum number of Connections a pool will maintain at any given time.
Default: 15.

\paragraph{c3p0.minPoolSize}
Minimum number of Connections a pool will maintain at any given time.
Default: 3.

\paragraph{c3p0.initialPoolSize}
Number of Connections a pool will try to acquire upon startup. Should be between
minPoolSize and maxPoolSize.
Default: 3.

\paragraph{c3p0.acquireIncrement}
Determines how many connections at a time c3p0 will try to acquire when the pool
is exhausted.
Default: 3.

\paragraph{c3p0.maxIdleTime}
Seconds a Connection can remain pooled but unused before being discarded. Zero
means idle connections never expire.
Default: 0.

\paragraph{c3p0.checkoutTimeout}
The number of milliseconds a client calling getConnection() will wait for a Connection
to be checked-in or acquired when the pool is exhausted. Zero means wait indefinitely.
Setting any positive value will cause the getConnection() call to time-out and
break with an SQLException after the specified number of milliseconds.
Default: 0.

\paragraph{c3p0.maxStatements}
The size of c3p0's global PreparedStatement cache. If both maxStatements and
maxStatementsPerConnection are zero, statement caching will not be enabled.
If maxStatements is zero but maxStatementsPerConnection is a non-zero value,
statement caching will be enabled, but no global limit will be enforced, only
the per-connection maximum. maxStatements controls the total number of Statements
cached, for all Connections. If set, it should be a fairly large number, as each
pooled Connection requires its own, distinct flock of cached statements.
Default: 0

\paragraph{c3p0.maxStatementsPerConnection}
The number of PreparedStatements c3p0 will cache for a single pooled Connection.
If both maxStatements and maxStatementsPerConnection are zero, statement caching
will not be enabled. If maxStatementsPerConnection is zero but maxStatements is
a non-zero value, statement caching will be enabled, and a global limit enforced,
but otherwise no limit will be set on the number of cached statements for a single
Connection. If set, maxStatementsPerConnection should be set to about the number
distinct PreparedStatements that are used frequently in application, plus
two or three extra so infrequently statements don't force the more common cached
statements to be culled.
Default: 0

\paragraph{c3p0.idleConnectionTestPeriod}
If this is a number greater than 0, c3p0 will test all idle, pooled but unchecked-out
connections, every this number of seconds.
Default: 0

%hibernate
\paragraph{hibernate.dialect}
The classname of a Hibernate org.hibernate.dialect.Dialect which allows to generate
SQL optimized for a particular relational database.

%hibernate search
\paragraph{hibernate.search.analyzer}
A fully qualified name of Lucene Analyzer that is used for the full text indexing
and searching. The default value is \emph{org.\-apache.\-lucene.\-analysis.\-cz.\-CzechAnalyzer}.
For more informations about Analyzers see Section \ref{sec:Lang}.

%NameTag
\subsubsection{Named Entity Recognizer Settings}
\label{sssec:NametagSettings}
\comment{Petr}{Add link to nametag?}
This section is mainly about settings stored in NametagLearning.properties. There
are parameters for neural network which nametag use for learning and switches for
nametag functions. Also, there are settings affecting models storing and
generating learning data. All inputs are assumed to be in UTF-8 encoding.

\paragraph{ner\_identifier}
Identifier of the named entity recognizer type. This affects the tokenizer used
in this model, and in theory any other aspect of the recognizer. Supported values
are: \emph{czech}, \emph{english}, \emph{generic}.

\paragraph{tagger}
Identifier of tagger. NameTag can utilize several taggers to obtain the tags and lemmas:

\begin{description}
\item[trivial]
Do not use any tagger. The lemma is the same as the given form and there is no
part of speech tag.
\item[external]
Use some external tagger. The input "forms" can contain multiple tab-separated
values, first being the form, second the lemma and the rest is part of speech tag.
The part of speech tag is optional. The lemma is also optional and if missing,
the form itself is used as a lemma.
\item[morphodita:model]
Use MorphoDiTa as a tagger with the specified model. This tagger model is embedded
in resulting named entity recognizer model. The lemmatizer model of MorphoDiTa
is recommended, because it is very fast, small and detailed part of speech tags
do not improve the performance of the named entity recognizer significantly.
\end{description}

\paragraph{featuresFile}
Path of generated file, based on property file.
The recognizer utilizes feature templates to generate features which are used as
the input to the named entity classifier. The feature templates are specified in
a file, one feature template on a line. Empty lines and lines starting with \# are ignored.

The first space-separated column on a line is the name of the feature template,
optionally followed by a slash and a window size. The window size specifies how
many adjacent words can observe the feature template value of a given word, with
default value of 0 denoting only the word in question.

List of commonly used feature templates follows. Note that it is probably not
exhaustive (see the sources in the features directory).

\comment{Ondrej}{It is necessary to add an example, ideally specific for police domain}

\paragraph{stages}
The number of stages performed during recognition. Common values are either 1 or
2. With more stages, the model is larger and recognition is slower, but more accurate.

\paragraph{iterations}
The number of iterations performed when training each stage of the recognizer.
With more iterations, training take longer (the recognition time is unaffected),
but the model gets over-trained when too many iterations are used. Values from 10
to 30 or 50 are commonly used.

\paragraph{missing\_weight}
Default value of missing weights in the log-linear model. Common values are small
negative real numbers like -0.2.

\paragraph{initial\_learning\_rate}
learning rate used in the first iteration of SGD training method of the log-linear
model. Common value is 0.1.

\paragraph{final\_learning\_rate}
learning rate used in the last iteration of SGD training method of the log-linear
model. Common values are in range from 0.1 to 0.001, with 0.01 working reasonably well.

\paragraph{gaussian}
The value of Gaussian prior imposed on the weights. In other words, the value of
L2-norm regularizer. Common value is either 0 for no regularization, or a small
real number like 0.5.

\paragraph{hidden\_layer}
Experimental support for hidden layer in the artificial neural network classifier.
To not use the hidden layer (recommended), use 0. Otherwise, specify the number
of neurons in the hidden layer. Please note that non-zero values will create
enormous models, slower recognition and are not guaranteed to create models with
better accuracy.

\paragraph{heldout\_data}
Optional parameter with heldout data in same format as training data. If the heldout data
is present, the accuracy of the heldout data classification is printed during
training. The heldout data is not used in any other way.

\paragraph{waiting\_time}
Maximum time for learning new model (only when learning is blocking - model doesn't exist)

\comment{Ondrej}{I do not understand, or I do, but common/unknowing reader would not}

\paragraph{default\_training\_data\_file}
Sample file with initial training data, which is used before training data
generated from database are available. When this property is empty, no default training data
are used.

\paragraph{maximum\_stored\_models}
Number of maximum stored models. When limit is exceeded, the oldest models are deleted.

\paragraph{form}
Size of window when forms are used as features (0 when they are not used)

\paragraph{lemma}
Size of window when lemma ids are used as features (0 when they are not used).
When lemma is specified, lemma IDs are used.

\paragraph{raw\_lemma}
Size of window when raw lemmas are used as features (0 when they are not used)
Difference between lemma and raw lemma is, thah lemma uses lemma IDs and raw\_lemma use raw lemmas as feature.

\paragraph{raw\_lemma\_capitalization}
Size of window when capitalization of raw lemma is used as features (0 when they are not used)

\paragraph{tag}
Size of window when tags are used as features (0 when they are not used)

\paragraph{numeric\_time\_value}
Recognize numbers which could represent hours, minutes, hour:minute time, days,
months or years.

\paragraph{czech\_lemma\_term}
Feature template specific for Czech morphological system by Jan Hajič (Hajič 2004).
The term information (personal name, geographic name, ...) specified in lemma comment
are used as features.

\paragraph{brown\_clusters}
Size of window foe Brown clusters \comment{Ondrej}{reference to book/article}.

\paragraph{brown\_clusters\_file}
Use Brown clusters found in the specified file. An optional list of lengths of
cluster prefixes to be used in addition to the full Brown cluster can be specified.
Each line of the Brown clusters file must contain two tab-separated columns,
the first of which is the Brown cluster label and the second is the raw lemma \comment{Ondrej}{What is it?}.

\paragraph{gazetteers}
Size of window when gazetteers are used (0 when are not used).

\paragraph{gazetteers\_directory}
Directory of gazetteers files. Each file is one gazetteer list independent
of the others and must contain a set of lemma sequences, each on a line,
represented as raw lemmas separated by spaces.

\paragraph{previous\_stage}
Use named entities predicted by previous stage as features.

\paragraph{email\_detector}
Detect URLs and emails. If an URL or an email is detected, it is immediately marked
with specified named entity type and not used in further processing.

\subsection{Data Formats}
To train a named entity recognizer model, training data is needed. The training data must be tokenized and contain annotated name entities. The name entities are non-overlapping, consist of a sequence of words and have a specified type.
and
The training data must be encoded in UTF-8 encoding. The lines correspond to individual words and an empty line denotes an end of sentence. Each non-empty line contains exactly two tab-separated columns, the first is the word form the second is the annotation. The format of the annotation is taken from CoNLL-2003: the annotation {\it I-type} or {\it B-type} denotes named entity of specified type, any other annotation is ignored. The {\it I-type} and {\it B-type} annotations are equivalent except for one case – if the previous word is also a named entity of same type, then
\begin{itemize}
\item{}if the current word is annotated as I-type, it is part of the same named entity as the previous word (Or begin of new entity, if previous word wasn't entity),
\item{}if the current word is annotated as B-type, it is in a different name entity than the previous word (albeit with the same type).
\end{itemize}
