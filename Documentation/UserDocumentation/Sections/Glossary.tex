Terms are used in very specific manner in the documentation and in both the
client and the server, therefore it is really essential to understand their
meaning. Otherwise, it is possible to misunderstand the rest of the text.

\comment[anyone]{Petr}{check if this makes sense!}
\begin{description}
\item[Document]
The document is an arbitrary text that is processed by \textan{}. E.g. a police
report, a letter, a movie review or a judgment. In the following text the terms
document and report are used interchangeably.

\item[Object]
The term "object" refers to our representation of some unique thing in
\textan{}. Consider for example a person with name Joseph Smith from Prague with
identifier 000123/4567. The object refers to this person in the real world, in
meat and bones.

\item[Alias]
The alias represents a designation of a object. For example: Joseph Smith, Joe
Smith, Joey S., J. Smith or just Smith can be names of one concrete person.
Names are aliases for the person.

\item[Entity]
The entity is a sequence of words recognized either by the named entity
recognizer or by a user. The entity will be later assigned to an object.
Consider for example this sentence: "Joe Smith was seen in Prague on 5th
September". The sentence contains 3 entities: Joe Smith (person), Prague (city),
5th September (date). We do not know which objects belong to these entities yet.
Prague can represent the capital city of the Czech Republic, or a city in the
USA.

\item[Object/Entity Type]
Each object and each entity has a type. For example person, date, city, country
etc. The set of types in the system depends on the domain which the system is
deployed to.

\item[Object Assignment]
\comment{Petr}{Need check.}
The Object assignment (or just assignment) is a process at which an object is
assigned to a recognized entity.  After the assignment, the text that represents
the entity in a document becomes the alias of the assigned object. The entity
and the assigned object must have the same type, this is the essential aspect of
the assignment. For example, there is a document in which an entity of type
"date" with text "14th March 2011" was found and there is an object in the
database of type "date" representing the day with aliases "the Fourteenth of
March, 2011", "14/3/2011" and "14/3/11". The object will be probably assigned to
the entity, because we know that they represent the same thing. After the
assignment, the object will also have the alias "14th March 2011".

\item[Alias Occurrence]
\comment{Petr}{How to describe this?}

\item[Object Merging/Joining]
\comment{Petr}{Need check.}
Because objects are created only based on information from documents, there may
be a situation when two objects represent same thing. The operation "merge/join
object" is used to solve this issue, it merges two objects into new one and
original objects are replaced with the new object. The new object has aliases of
both original objects. The reverse operation is called "split objects".

\item[Relation]
The relation represents any relationship between objects mentioned in a
document. The relation can be represented in the document by an anchor word or
phrase, or it can be unexpressed, which means without any anchor in the
document. See Anchor below.
Consider this example: "Adam (890524/1367) is married to Eve". Three objects
appear in the sentence - persons Adam and Eve and the id 890524/1367. There
are two relations: Adam "has" 890524/1367 without an anchor and Adam "marriage"
Eve with the anchor "married".

\item[Relation Type]
Each relation has a type. For example born, reside, die, kill etc. Similarly to
object/relation types the set of relation types in the system depends on the
domain which the system is deployed to.

\item[Anchor]
The anchor is a word or a sequence of words that is bound to a relation between
objects. Consider for example the sentence: Joey Smith killed Mary-Ann S. There
is a relation between Joey Smith and Mary-Ann S., which is represented by the
word "killed". The word is the anchor.
\comment{Ondrej}{Why do you not write about type here?}
\comment{Petr}{The example should show what is an anchor. The type isn't important in this situation.}

\item[Roles and orders]
Each object in a relation can have a role assigned. This role is mostly used
for visualization and storing additional information. Moreover object can also
have order assigned which is just an integral number used for graph
visualization\ifdefined\USRDOC{} (see Section \ref{sssec:EditRelations})\fi{}
\ifdefined\DEVDOC{} (see Section \ref{USR-sssec:EditRelations} in the user
documentation)\fi{}. For example relation created from sentence
"Bill killed Joe." could be represented like this:
Bill (-1, murderer) --kill--> (1, victim) Joe.

\item[Relation Occurence]
\comment{Petr}{How to describe this?}
\end{description}

