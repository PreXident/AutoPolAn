Terms are used in very specific manner in the documentation and in both applications,
therefore it is really essential to understand their meaning. Otherwise, it is
possible to misunderstand the rest of the text.

\comment[anyone]{Petr}{check if this makes sense!}
\begin{description}
\item[Object]
The term "object" is a representation of some unique thing in \textan{}. Consider
for example a person with name Joseph Smith from Prague with identifier 000123/4567.
The object refers to this person in the real world, a meat and bones.

\item[Alias]
The alias represents a designation of a object. For example: Joseph Smith, Joe
Smith, Joey S., J. Smith or just Smith can be names of one concrete person.
Names are aliases for the person.

%\item[Alias occurrence]

\item[Document]
The document is an arbitrary text that is processed by \textan{}. E.g. a police
report, a letter, a movie review or a judgment. In the following text the terms
document and report are used interchangeably.

\item[Entity]
The entity is a sequence of words recognized either by a named entity recognizer
or by a user. The entity will be later assigned to an object and it became an alias
of the object after assigning. Consider for example a sentence: Joe Smith was seen
in Prague on 5th September. The sentence contains 3 entities: Joe Smith (person),
Prague (city), 5th September (date). We don't know which objects belongs to these
entities at the moment. Prague can represent the capital city of the Czech Republic,
or a city in the USA.

\item[Object/Entity Type]
Each object and entity has a type. For example person, date, city, country etc.
Set of types in the system depends on domain which the system is deployed to.
There are some universal predefined ones, but the system administrator needs to
prepare customized list to fully unleashed the power of the project.

\item[Relation]
The relation represents any relationship between objects mentioned in a document.
A relation can be represented in a document by an anchor, or can be unexpressed,
which means without any anchor in a document. See Anchor below.
\comment[anyone]{Adam}{Check this sentence with two relations, one with anchor, one without anchor.}
Consider this example: "Adam (890524/1367) is married to Eve". Three objects
have occurrence in the sentence - persons Adam and Eve and id 890524/1367. There
are two relations: Adam "has" 890524/1367 without anchor and Adam "marriage" Eve
with anchor "married".

\item[Relation Type]
Each relation has a type. For example born, reside, die, kill etc. Similarly to
object/relation types set of relation types in the system depends on domain
which the system is deployed to. There are some universal predefined ones, but
the system administrator needs to prepare customized list to fully unleashed the
power of the project.

\item[Anchor]
The anchor is a word or a sequence of words that is binded to a relation between
objects. Consider for example a sentence: Joey Smith killed Mary-Ann S. There is
relation between Joey Smith and Mary-Ann S., which is represented by the word
"killed". The word is the anchor.

\item[Roles and orders]
Each object in a relation can have a role assigned. This role is mostly used
for visualisation and storing additional information. Moreover object can also
have order assigned which is just an integral number used for graph
visualization\ifdefined\USRDOC{} (see \ref{sssec:EditRelations})\fi{}
\ifdefined\DEVDOC{} (see user documentation section
\ref{USR-sssec:EditRelations})\fi{}. For example relation created from sentence
"Bill killed Joe." could be represented like this:
Bill (-1, murderer) --kill--> (1, victim) Joe.
\end{description}

\comment{Petr}{is it necessary to explain terms: assignment, object merging, alias occurence?}
\comment{Adam}{Let's wait for Ondrej's feedback}
