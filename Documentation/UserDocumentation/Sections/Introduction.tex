% Section: Introduction


This section is intended to provide introduction to the project and glossary of
terms used in the rest of the documentation and the application itself.

\subsection{System overview}

With an increasing amount of electronically distributed unstructured text (newspapers,
letters etc.) there is growing demand for its analyzing and understanding computer.
This process should be automated due to the time consumption and huge data amount.

\comment{Petr}{Remove mentions of police.}

\textan{} (Text Analyser) serves such purpose, targeting the Police reports. 
In other words, \textan{} is a tool which supports mining structured data buried
in the Police reports. The term ``structured data'' consists of two concepts. 
First, it contains name, street, date of birth, crime and other named entities related to some people. 
Second, the data contain the relations between two entities. 
For example, the relation between a person and their date of birth, the relation between two persons. The objective of \textan{} is to provide a robust tool which supports the procedure, either automatically and manually.
Although first impulse came from the Police needs, \textan{} is made as general
as possible to allow easy configuration for any other domain as needed.

\textan{} has client-server structure. It supports following operations: \comment{Petr}{Operations!?}
  
\begin{itemize}
  \item Bring new solution to the classic problem of extracting data from text.
  \item Provide the service for both automatic detection and manual adjustment.
  \item Make the adjustment of data as simple as possible for users.
  \item Make the graphic user interface as fruitful as possible.
\end{itemize}

This documentation describes \textan{} from the perspective of an user. 
All the concepts are explained in the \emph{Glossary}. \comment{Petr}{useless}

\subsection{Glossary}
Terms are used in very specific manner in the documentation and in both the
client and the server, therefore it is really essential to understand their
meaning. Otherwise, it is possible to misunderstand the rest of the text.

\comment[anyone]{Petr}{check if this makes sense!}
\begin{description}
\item[Document]
The document is an arbitrary text that is processed by \textan{}. E.g. a police
report, a letter, a movie review or a judgment. In the following text the terms
document and report are used interchangeably.

\item[Object]
The term "object" refers to our representation of some unique thing in
\textan{}. Consider for example a person with name Joseph Smith from Prague with
identifier 000123/4567. The object refers to this person in the real world, in
meat and bones.

\item[Alias]
The alias represents a designation of a object. For example: Joseph Smith, Joe
Smith, Joey S., J. Smith or just Smith can be names of one concrete person.
Names are aliases for the person.

\item[Entity]
An entity is a sequence of words recognized either by a named entity recognizer
or by a user. \comment[Petr]{Petr}{Move to next paragraph - Object assigment}
The entity will be later assigned to an object and it becomes an alias
of the object. Consider for example the sentence: Joe Smith was seen
in Prague on 5th September. The sentence contains 3 entities: Joe Smith (person),
Prague (city), 5th September (date). We don't know which objects belong to these
entities yet. Prague can represent the capital city of the Czech Republic,
or a city in the USA.

\item[Object/Entity Type]
Each object and each entity has a type. For example person, date, city, country
etc. The set of types in the system depends on the domain which the system is
deployed to.

\item[Object Assigment]
\comment[Petr]{Petr}{Description of object assigment}

\item[Alias occurrence]
\comment[Petr]{Petr}{Description of alias occurence}

\item[Object Merging]
\comment[Petr]{Petr}{Description of object merging}

\item[Relation]
The relation represents any relationship between objects mentioned in a document.
A relation can be represented in a document by an anchor word, or it can be
unexpressed, which means without any anchor in a document. See Anchor below.
\comment[anyone]{Adam}{Check this sentence with two relations, one with anchor, one without anchor.}
Consider this example: "Adam (890524/1367) is married to Eve". Three objects
appear in the sentence - persons Adam and Eve and the id 890524/1367. There
are two relations: Adam "has" 890524/1367 without an anchor and Adam "marriage"
Eve with the anchor "married".

\item[Relation Type]
Each relation has a type. For example born, reside, die, kill etc. Similarly to
object/relation types the set of relation types in the system depends on the
domain which the system is deployed to.

\item[Anchor]
An anchor is a word or a sequence of words that is bound to a relation between
objects. Consider for example the sentence: Joey Smith killed Mary-Ann S. There
is a relation between Joey Smith and Mary-Ann S., which is represented by the
word "killed". The word is the anchor. \comment{Ondrej}{Why do you not write
about type here?}

\item[Roles and orders]
Each object in a relation can have a role assigned. This role is mostly used
for visualisation and storing additional information. Moreover object can also
have order assigned which is just an integral number used for graph
visualization\ifdefined\USRDOC{} (see Section \ref{sssec:EditRelations})\fi{}
\ifdefined\DEVDOC{} (see Section \ref{USR-sssec:EditRelations} in the user
documentation)\fi{}. For example relation created from sentence
"Bill killed Joe." could be represented like this:
Bill (-1, murderer) --kill--> (1, victim) Joe.
\end{description}


