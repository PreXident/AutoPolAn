% Section: Language

Although \textan{} is aimed particularly on the Czech language and is preset for
it by default, it is possible to configure \textan{} for other languages, especially
English. There are two areas which depend on the selected language: named entity
recognition and full text searching. Both are placed in the server application
same as other business logic.

\comment[Kuba]{Petr}{Here should be some summary about changing language in nametag}
NameTag is implemented to support more languages, so if you want use {\it NameTag} in different language, you must change property, which set language  - {\it ner\_identifier},
specify appropriate tagger for required language and provide data in desired language.
See \ref{sssec:NametagSettings} for detailed description.

The second thing, that needs to be configured, is the full text search. The system
internally use Apache Lucene, full text search engine. Lucene provides creation
of full text indexes and searching in them. To get good results, it is necessary
perform some preprocessing, such as replacing words with their stems (stemming),
stop words filtering or synonym expansion. This is done through an Analyzer. Lucene
contains Analyzers\footnote{\url{http://lucene.apache.org/core/4_7_0/analyzers-common/overview-summary.html}}
for many common languages, or you can define your own. Selection of Analyzer in 
\textan{} is done through property \emph{hibernate.search.analyzer} in \emph{data.properties}.
See \ref{sssec:DataSettings}.

