% Section: Language

Although \textan{} particularly aim at processing Czech language and is preset for
Czech language by default, it is possible to configure \textan{} for other languages,
especially English. However, \textan{} supports only one language at a time. There are
two areas which depend on the selected language: named entity recognition and full
text searching. Both of them are placed in the server application along with other
business logics.

The first thing that needs to be re-configured for new language is {\it NameTag}. 
You must change property {\it ner\_identifier}
which affects it, specify appropriate tagger and provide
training data in desired language. See Section \ref{sssec:NametagSettings} for
detailed description.

The second thing, that needs to be configured, is the full text search. The system
internally uses Apache Lucene, full text search engine. Lucene provides creation
of full text indexes and searching in them. To get good results, it is necessary
to perform some preprocessing, such as replacing words with their stems
(stemming), stop words filtering or synonym expansion. This is done through an
Analyzer. Lucene contains Analyzers%
\footnote{\url{http://lucene.apache.org/core/4_7_0/analyzers-common/overview-summary.html}}
for many common languages, or you can define your own. Selection of Analyzer in 
\textan{} is done through property \emph{hibernate.search.analyzer} in
\emph{data.properties}. See Section \ref{sssec:DataSettings}.
