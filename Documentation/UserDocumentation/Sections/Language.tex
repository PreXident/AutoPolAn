% Section: Language

Although \textan{} was developed with Czech  as the first supported language
and it is preset for
Czech language by default, it is possible to configure \textan{} for other languages,
especially English. For now, \textan{} assumes that all documents in the dataase
are in a single language. There are
two areas which depend on the selected language: named entity recognition and full
text search. Both of them are placed in the server application along with other
business logic.

The first thing that needs to be re-configured for a new language is {\it
NameTag}, the name entity recognizer. 
You must change property {\it ner\_identifier}
specifying an appropriate tagger and providing
training data in desired language. See Section \ref{sssec:NametagSettings} for
a detailed description.

The second thing, that needs to be configured, is the full text search. The system
internally uses Apache Lucene, full text search engine. Lucene provides creation
of full text indexes and searching in them. To get good results, it is necessary
to perform some preprocessing, such as replacing words with their stems
(stemming), removing stop words or expanding synonyms. This is done through an
Analyzer. Lucene contains Analyzers%
\footnote{\url{http://lucene.apache.org/core/4_7_0/analyzers-common/overview-summary.html}}
for many common languages, or you can define your own. The selection of an Analyzer in 
\textan{} is done through the property \emph{hibernate.search.analyzer} in
\emph{data.properties}. See Section \ref{sssec:DataSettings}.
