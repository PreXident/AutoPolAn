% Section: Client

This section contains all information needed to successfully install, configure
and run \textan{} JavaFX client side application. However thanks to \textan{}
client-server architecture and extensive use of webservices, it is easy to
prepare own implementation in any programming language desired or even use
general solutions as SoapUI%
\footnote{More information can be found at \url{http://www.soapui.org/}},
if \textan{} default client does not provide all required features, user
experience or if some integration to a legacy system is needed.

\subsection{Installation guide}

This section describes how to install and configure \textan{} client on
workstations of end users.

\subsubsection{Prerequisites}

The \textan{} client has the same requirement for \emph{Java 8 JRE} or later
as the server (see \ref{sssec:SerInstPre}). However, unlike the server side,
it is a pure Java application so there are no other additional limitations for
operating system. Any platform where JRE 8 is available can support the
client.

\subsubsection{Installation}

Unpack an archive with the \textan{} client distribution into any directory. 
The archive contains a .jar file and starting scripts for both
\emph{Microsoft Windows} and \emph{Linux-based} operating systems
\footnote{run.bat for MS Windows and run.sh for Linux clones.\label{runscript_note}}.

\subsubsection{Basic configuration}
\label{sssec:BasicConf}

If the client is not supposed to connect to the default server, it is necessary
to configure a server location. The client settings are by default stored in
\emph{TextAn.properties\footnote{A description of format of .properties file can
be found at \url{http://en.wikipedia.org/wiki/.properties}}}
in the client directory. Simply edit or add following lines to properties file
(create it if it does not exist), replacing default server address with the
actual one:
\begin{lstlisting}[frame=single,language=properties]
#url of the document processor
url.document=http://localhost:9500/soap/document
#url of the document processor wsdl
url.document.wsdl=http://localhost:9500/soap/document?wsdl
#url of the data provider
url.data=http://localhost:9500/soap/data
#url of the data provider wsdl
url.data.wsdl=http://localhost:9500/soap/data?wsdl
\end{lstlisting}

It is also possible to set timeouts for communication with server, if default
values (60,000 ms) are not suitable. Specify -1 for disabling timeout.
\begin{lstlisting}[frame=single,language=properties]
#timeout for connecting in ms (default 60000)
connect.timeout=10000
#timeout for request in ms (default 60000)
request.timeout=10000
\end{lstlisting}

If SSL is in use, it needs to be configured by setting following properties:
\begin{lstlisting}[frame=single,language=properties]
#should ssl be used for communication?
ssl=true
#path to trust store
ssl.trustStore=c\:/temp/clientTrustStore
#trust store password
ssl.trustStore.password=MYPASS
#trust store type
ssl.trustStore.type=JKS
\end{lstlisting}

If client authorization is in use along with SSL, it needs to be configured
by settings following properties:
\begin{lstlisting}[frame=single,language=properties]
#should client authorization be used for communication?
ssl.clientAuth=true
#path to key store
ssl.keyStore=c\:/temp/clientKeyStore
#key store password
ssl.keyStore.password=MYPASS
#key store type
ssl.keyStore.type=JKS
\end{lstlisting}

For more information about security, please consult \ref{sec:Security}.

\subsubsection{Starting the client}
\label{sssec:StartClient}

The .jar file from the distribution archive is executable,
but we recommend to use starting scripts\footref{runscript_note}.
\comment{Ondrej}{This is confusing, when on next page, use ``see note X on page Y''}
Please consult the documentation of your Java Platform provider,
if running scripts are not available for your system.
For description of command line arguments see \ref{ssec:CliCmdArg}

\subsubsection{Uninstallation}

The client application does not use any files or resources outside its install
directory, unless explicitly told to do so.
To uninstall the client just delete its installation folder and all files
explicitly used as configuration storage.

\subsection{Settings}

This section describes all settings affecting the client behaviour.
The two basic means are command line arguments and configuration file in
properties format.

\subsubsection{Command line arguments}
\label{ssec:CliCmdArg}

The client has one command line option (-h, -{}-help, /H, /?) which displays the
usage information. Apart from that it takes only one command line argument
which is the location of settings file. If no file is specified, the default
property file \emph{TextAn.properties} will be used. If '-' is provided, the
standard input will be read and settings will not be stored.

\subsubsection{TextAn.properties}

For complete list of properties stored in TextAn.properties, please see
Developer Documentation apendix Client Properties in TextAn.properties.

\comment[latex expert, Peter?]{Adam}{Please, someone more skilled in latex use
proper way to reference the other document. Thanks.}

For more information about \emph{clear.filters}, \emph{graph.distance},
\emph{hypergraphs}, \emph{locale.language}, \emph{username}
and \emph{windows.independent} please see \ref{sssec:GeneralSettings}.
