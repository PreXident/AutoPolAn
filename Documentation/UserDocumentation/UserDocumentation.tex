\documentclass[12pt,a4paper]{report}

\usepackage[utf8]{inputenc}
\usepackage[english]{babel}

\usepackage{geometry}
\usepackage{xcolor}
\usepackage{amsmath}
\usepackage[some]{background}
\usepackage{listings}
\usepackage{footmisc}

%TODO select color coresponding to logo
\definecolor{titlepagecolor}{cmyk}{1,.60,0,.40}
\definecolor{javagreen}{rgb}{0.25,0.5,0.35}

\backgroundsetup{
scale=1,
angle=0,
opacity=1,
contents={
\begin{tikzpicture}[remember picture,overlay]
  \path [fill=titlepagecolor] (current page.west)rectangle (current page.north east); 
  \draw [color=white, very thick] (5,0)--(5,0.5\paperheight);
 \end{tikzpicture}}
}

\DeclareFixedFont{\bigsf}{T1}{phv}{b}{n}{1.5cm}

\usepackage[unicode,colorlinks=true]{hyperref}
\hypersetup{pdftitle=TextAn - user documentation}
\hypersetup{pdfauthor={Petr Fanta, Duc Tam Hoang, Adam Huječek, Václav Pernička, Jakub Vlček}}
\hypersetup{linkcolor=black, citecolor=black, urlcolor=black, filecolor=black}

\makeatletter
\def\@makechapterhead#1{
  {\parindent \z@ \raggedright \normalfont
   \Huge\bfseries \thechapter. #1
   \par\nobreak
   \vskip 20\p@
}}
\def\@makeschapterhead#1{
  {\parindent \z@ \raggedright \normalfont
   \Huge\bfseries #1
   \par\nobreak
   \vskip 20\p@
}}
\makeatother

\def\chapwithtoc#1{
\chapter*{#1}
\addcontentsline{toc}{chapter}{#1}
}

\setcounter{secnumdepth}{3}
\setcounter{tocdepth}{3}

\lstdefinelanguage{properties}{% new language for listings
  basicstyle=\ttfamily,
  sensitive=false,
  morecomment=[l]{\#},      % comment
  morestring=[b]",          % string def
  commentstyle=\color{javagreen},
  basicstyle=\small
}

\setkeys{Gin}{resolution=96}

\newcommand{\textan}{\emph{TextAn}}

\begin{document}

%TODO better titlepage
\begin{titlepage}
\BgThispage

\newgeometry{left=2cm,right=6cm,bottom=2cm}
\vspace*{0.3\textheight}
\noindent
\textcolor{white}{\bigsf TextAn}

\vspace*{1cm}
\noindent
\textcolor{white}{\Huge\textbf{\textsf{User Documentation}}}

\vspace*{2cm}\par
\noindent
\begin{minipage}{0.35\linewidth}
\textbf{Authors} \\
Petr Fanta \\
Duc Tam Hoang \\
Adam Huječek \\
Václav Pernička \\
Jakub Vlček\vspace{40pt} \\
\textbf{Supervisor} \\
Ondřej Bojar\vspace{40pt} \\
\textbf{Version} \\
0.1\vspace{40pt} \\
\textbf{Date} \\
\today \\
\end{minipage}

%TODO add logo

\end{titlepage}
\restoregeometry

\pagenumbering{roman}
\tableofcontents

%\chapter*{Intro}
%\addcontentsline{toc}{chapter}{Intro}


\chapter{User Guide}
\pagenumbering{arabic}

\section{Introduction}

% FOR ENTERTAIN: The spider opens his heavy eyes. He stand up, look through the lousy windows. The sun is flaring, kissing all over the earth. He said to himself: ``It's time''. Students are rushing out of the dormitory like rats desert from a sinking ship. Some are going to the public transport, waiting for the regular hooves from the distance. The bus comes and goes, its double tires sing high notes on the road. Some are visiting the car park. The wheels of the cars creaked around, then they crawled into the city centre. A man and girl are crossing street, with their arms around each other's waists. Then the man must have said something supposed to be funny because two of them are laughing like hyenas. The spider retreats from the windows. The daily screen makes him so lonesome and depressed. Another day has just begun.


%TODO what is textan

%TextAn (Text Analyser) is the product of our software project groups (consists of 5 members) at Faculty of Mathematics and Physics, Charles University in Prague (MFF UK). The tool is attributed to Software Project subject, developed in 9 months from Jan 2014 to Sep 2014, and supervised by RNDr. Ondřej Bojar, PhD. It is a client-server tool which support mining structured information from text documents. At the moment, the documents are specified to police report but TextAn could be applied to a wide range of domains. 

% --> to DeveloperDocumentation
% 

%TODO basic usage


\subsection{System overview}

%What is this?

%what textan do, something which is 

%WHAT DOES THE POLICE WANT? OF COURSE NOT ARREST US

% 

% it should be something better, for example, what textan to in a : What textan to. what you can expect in the rest of documentation . If you don't need what we offer in textan, leave a comment and get away.



With the profiliferation of unstructured written texts, the need for a tool to mine the structure out of such documents is on increasing trend. 
TextAn (Text Analyser) serves such purpose, targeting the Police's reports. 
In other words, TextAn is a tool which supports mining structured data buried in the Police's reports. The term ``structured data'' consists of two concepts. 
First, it contains name, street, date of birth, crime and other named entities with related to some people. 
Second, the data contain the relations between two entities. 
For example, the relation between a person and sis/her name, the relation between two persons. The objective of TextAn is to provide a robust tool which support the procedure, either automatically and manually.

TextAn has client-server structure. It supports following operations:
  
  \begin{itemize}
  \item Bring new solution to the classic problem of extracting data from text.
  \item Provide the service for both automatic detection and manual adjustment.
  \item Make the adjustment of data as simple as possible for users.
  \item Make the graphic user interface as fruitful as possible.
  \end{itemize}

This documentation describes TextAn from the perspective of an user. 
All the concepts are explained in the \texbf{Glossary}.

To start using TextAn, user have to start the client application and log in. 
Once user log in successfully, the main screen of TextAn is displayed.

%TODO: What's next?

\subsection{Glossary}

TODO definitions of terms in documentation and applications

\subsection{Conventions} %is this section needed?

\section{Client Usage}

\subsection{First run}

\subsection{Working space}

\subsection{Settings}
\label{ssec:Settings}

\subsection{Report processing}

\subsection{Listing objects}

\subsection{Graph views}

\subsection{Joining objects}

\chapter{Administrator guide}

%TODO client instalation guide in the administrator guide or user guide?

\section{Server}

\subsection{Installation guide}


This section will show you how to install and build \textan\ server. 
%An installation guide for a server side application of \textan.

\subsubsection{Prerequisites}
\label{sssec:SerInstPre}

The \textan\ server requires installation of \emph{Java 8 JRE
\footnote{We recommend to use JRE from \emph{Oracle}, available at: \url{http://www.oracle.com/technetwork/java/javase/downloads/index.html}}}
or a later version and a relational database system (e.g. \emph{MySQL}).
The server should be platform independent, so it runs on any system where JRE is available,
but it depends on native and 3rd party libraries. 
Supported operating systems in the distribution of the \textan\ server are
Linux (32 bit and 64 bit) and Windows (32 bit and 64 bit). % I am thinking ...

\subsubsection{Installation}

The \textan\ server distribution archive contains all files needed for \textan\ server to operate,
such as server binaries, native libraries, run scripts and additional resources.
For installation it is sufficient to unpack its content into any directory.

\subsubsection{Basic configuration}

Before the first start, it is necessary to set up the server and the database.

TODO write how to setup server

TODO add some link to \ref{sec:ServerSettings}

\subsubsection{Starting the server}

The server can be started by starting scripts in its root directory. 
Scripts for different operating systems unfortunately do not provide the same functionality.

The start script for the Linux-based operating system (\emph{run.sh}) runs the server application as daemon
and can be used to run the server as system service.

The start script for the Microsoft Windows OS is less powerful, only runs the server application. 
To run the server as a background process, the Windows Service is required. 
Unfortunately, there are no components shipped with the Jetty Distribution to make it a formal Windows Service.
However, we recommend the use of \emph{Apache ProcRun's Daemon
\footnote{More information can be found on \url{https://commons.apache.org/proper/commons-daemon/procrun.html} and in JavaDoc documentation for \emph{cz.cuni.mff.ufal.textan.server.AppEntry} class.}}.

\subsection{Settings}
\label{sec:ServerSettings}

\subsubsection{Web server settings}

%connector
\paragraph{server.connector.host} The particular interface to listen on. If not set or 0.0.0.0, the web server listens on port on all interfaces.

\paragraph{server.connector.port} The port to listen on. If not set, the web server listens on port 9500.

%thread pool
\paragraph{server.threadPool.maxThreads} The maximum number of threads in web server thread pool. It determines a maximum number of simultaneously opened connections. The default value is 200.

\paragraph{server.threadPool.minThreads} The minimum number of threads in web server thread pool. The default value is 8.

\paragraph{server.threadPool.idleTimeout} The time in milliseconds that the connection can be idle before it is closed.

\subsubsection{Database connection settings}

%jdbc
\paragraph{jdbc.driverClassName} 

\paragraph{jdbc.url} 

\paragraph{jdbc.user}

\paragraph{jdbc.pass}


%c3p0 (http://www.mchange.com/projects/c3p0/#configuration_properties)
\paragraph{c3p0.maxPoolSize}

\paragraph{c3p0.minPoolSize}

\paragraph{c3p0.initialPoolSize}

\paragraph{c3p0.acquireIncrement}

\paragraph{c3p0.maxIdleTime}

\paragraph{c3p0.checkoutTimeout}

\paragraph{c3p0.maxStatements}

\paragraph{c3p0.maxStatementsPerConnection}

\paragraph{c3p0.idleConnectionTestPeriod}

%hibernate
\paragraph{hibernate.dialect}

\paragraph{hibernate.show\_sql}

\subsubsection{Named entity recognizer settings}

\subsection{Data formats}

\section{Client}

This section contains all information needed to successfully install, configure and run \textan\ client side application.
However thanks to \textan\ client-server architecture and extensive use of webservices,
it is easy to prepare own implementation in any programmer language desired or even use general solutions as SoapUI
\footnote{More information can be found at \url{http://www.soapui.org/}},
if \textan\ default client does not provide all required features, user experience or integration to legacy system is needed.

\subsection{Installation guide}

This section describes how to install and configure \textan\ client on workstations of end users.

\subsubsection{Prerequisites}

The \textan\ client has the same requirement for \emph{Java 8 JRE} or later as the server (see \ref{sssec:SerInstPre}).
However, unlike the server side, it is pure Java application so there are no other additional limitations for operating system etc.
Any platform where JRE 8 is available can support the client.

\subsubsection{Installation}

Unpack an archive with the \textan\ client distribution into any directory. 
The archive contains a .jar file and starting scripts for the \emph{Microsoft Windows} and Linux-based operating systems
\footnote{run.bat for MS Windows and run.sh for Linux clones.\label{runscript_note}}.

\subsubsection{Basic configuration}

If the client is not supposed to connect to the default server, it is necessary to configure a server location. 
The client settings are by default stored in \emph{TextAn.properties\footnote{A description of format of .properties file can be found at \url{http://en.wikipedia.org/wiki/.properties}}}
in the client directory. 
Simply edit or add following lines to properties file (create it if it does not exist),
replacing default server address with the actual one:
\begin{lstlisting}[frame=single,language=properties]
#url of the document processor
url.document=http://localhost:9500/soap/document
#url of the document processor wsdl
url.document.wsdl=http://localhost:9500/soap/document?wsdl
#url of the data provider
url.data=http://localhost:9500/soap/data
#url of the data provider wsdl
url.data.wsdl=http://localhost:9500/soap/data?wsdl
\end{lstlisting}

\subsubsection{Starting the client}

The .jar file from the distribution archive is executable,
but we recommend to use starting scripts\footref{runscript_note}.
Please consult documentation of your Java Platform provider,
if running scripts are not available for your system.
For description of command line arguments see \ref{ssec:CliCmdArg}

\subsubsection{Uninstallation}

The client application does not use any files or resources outside its install directory,
unless explicitly told to do otherwise.
To uninstall the client just delete its installation folder
and all files explicitly used as configuration storage.

\subsection{Settings}

This section describes all settings affecting the client behaviour.
The two basic means are command line arguments and configuration file in properties format.

\subsubsection{Command line arguments}
\label{ssec:CliCmdArg}

The client has one command line option (-h, --help, /H, /?) which displays the usage information.
Apart from that it takes only one command line argument which is the location of settings file.
If no file is specified, default property file \emph{TextAn.properties} will be used.
If '-' is provided, the standard input will be read and settings will not be stored.

\subsubsection{TextAn.properties}

This section contains a list of properties from configuration file that controls
client behaviour and their brief description.

\begin{lstlisting}[frame=single,language=properties]
#main application window height
application.height=600.0
#is main application window maximized?
application.max=true
#main application window width
application.width=800.0
#main application window x-pos
application.x=192.0
#main application window y-pos
application.y=119.0
#indicator whether the filters in context menus in pipeline
#should be cleared when item is selected
clear.filters=true
#document edit/add window height
document.edit.height=300.0
#is document edit/add window maximized?
document.edit.maximized=true
#document edit/add window width
document.edit.width=450.0
#document edit/add window x-pos
document.edit.x=193.0
#document edit/add window y-pos
document.edit.y=164.0
#document view window height
document.viewer.height=576.0
#is document view window height maximized?
document.viewer.maximized=true
#document view window widht
document.viewer.width=662.0
#document view window x-pos
document.viewer.x=148.0
#document view window y-pos
document.viewer.y=203.0
#number of documents to list on one page in document
#view window
documents.per.page=100
#document list window height
documents.viewer.height=604.0
#is document list window maximized?
documents.viewer.maximized=true
#document list window width
documents.viewer.width=663.0
#document list window x-pos
documents.viewer.x=513.0
#document list window y-pos
documents.viewer.y=87.0
#default distance from graph center to display
graph.distance=6
#graph window height
graph.viewer.height=784.0
#is graph window maximized?
graph.viewer.maximized=true
#graph window width
graph.viewer.width=678.0
#graph window x-pos
graph.viewer.x=60.0
#graph window y-pos
graph.viewer.y=26.0
#flag indicating whether the hypergraphs should be displayed
#as background color instead of by additional vertex
hypergraphs=true
#object join window height
join.view.height=682.0
#is object join window maximized?
join.view.maximized=true
#object join window width
join.view.width=953.0
#object join window x-pos
join.view.x=147.0
#object join window y-pos
join.view.y=91.0
#directory lastly used for save/load a report
loadreport.dir=C\:\\temp
#application language
locale.language=cs
#number of documents to list on left page
#in object join window
objects.per.page.left=100
#number of documents to list on right page
#in object join window
objects.per.page.right=100
#number of documents to list on one page
#in object list window
objects.per.page=100
#relation list window height
relation.view.height=741.0
#is relation list window maximized
relation.view.maximized=true
#relation list window width
relation.view.width=722.0
#relation list window x-pos
relation.view.x=149.0
#relation list window y-pos
relation.view.y=15.0
#report wizard pipeline window height
report.wizard.height=734.0
#is report wizard pipeline window maximized?
report.wizard.maximized=true
#report wizard pipeline window width
report.wizard.width=843.0
#report wizard pipeline window x-pos
report.wizard.x=397.0
#report wizard pipeline window y-pos
report.wizard.y=70.0
#report wizard pipeline window height
selectfile.dir=C\:\\temp
#files with extension 'txt' should be processed as text files
#encoded in Windows-1250 encoding only other valid value for
#now is TEXT_UTF8 for utf-8 encoding
selectfile.extension.txt.type=TEXT_CP1250
#url of the data provider wsdl
url.data.wsdl=http\://localhost\:9500/soap/data?wsdl
#url of the data provider
url.data=http\://localhost\:9500/soap/data
#url of the document processor wsdl
url.document.wsdl=http\://localhost\:9500/soap/document?wsdl
#url of the document processor
url.document=http\://localhost\:9500/soap/document
#user's login
username=BFU
#flag indicating whether independent system windows should
#be used instead of embedded inner windows
windows.independent=false
\end{lstlisting}

For more information about \emph{clear.filters}, \emph{graph.distance},
\emph{hypergraphs}, \emph{locale.language}, \emph{username}
and \emph{windows.independent} please see \ref{ssec:Settings}

\end{document}
