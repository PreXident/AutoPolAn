%This section contains work progress
In this section is described work progress in each month.

\subsection{October}
\comment[anyone]{Adam}{Add proper rank of Honza}\\
TextAn team visited policeman Jan Hořínek, who introduced us his work and showed
us existing inadequate software whose extension we should prepare. Basis of the
project should be processing police reports - recognize entities and match them
to existing objects in the database. We should use Webservices to provide easy
way to integrate the project to the current system. He promised us models and
example data inputs and expected outputs. We divided the work as follows:

\begin{itemize}
\item \textbf{Petr Fanta} - entities recognition and input parsing, look for useful existing solutions
\item \textbf{Adam Huječek} - GUI validation, component schema and communication
\item \textbf{Václav Pernička} - database
\item \textbf{Peter Šípoš} - web service, graph GUI, testing
\item \textbf{Jakub Vlček} - entities recognition and input parsing, look for useful existing solutions
\end{itemize}

We made some decisions and requirements for each component:

\paragraph{Database}
\comment{Jakub}{attach document schema here?}
At the beginning, we made database schema, which was approved after a few
iterations. We paid attention mainly on schema generality because we want to
support more than one (police) domain.\\
Database should support versioning, parallel processing, partial records. We
decided to create special layer between server and database because of
generality (it will be easier to change database system). Logging (users,
changes) support should not be missing. Problem could be with merging two
objects into new one, so we should create special functionality and table in the
database.

\paragraph{Client}
We decided for pipeline model. This means that report processing will consist of more successive phases:

\begin{enumerate}
\item Report insertion
\item Report editing
\item (Auto) Entity recognition
\item Entity editing (entity types, ranges, creating new ones)
\item (Auto) Object recognition
\item Object editing (adding new ones, repair bad connections between entities and objects)
\item (Auto) Relationships recognition
\item Relationships editing
\item Report confirmed and sent to server
\end{enumerate}

Changes in each phase are not stored in the database immediately but they are
stored locally and write to database follows confirmation. Problem is with
machine recognition, because models are not trained to newly added items, so
they are preferred over recognized ones.
\comment[Jakub]{Adam}{What does the last sentence mean?}
 
\subsection{November}
In November, we mainly chose and tested technologies which we could use. Peter
Sipos left \textan{}, but Duc Tam Hoang joined us. His specialization is
linguistic and machine learning so he will be really useful for our project.
Because he is Vietnamese student we had to changed project language to English.
His main task will be NER and entity-object matching. We were trying different
entity recognizers, but there are not many applicable for Czech. Milan Straka
promised us his project NameTag. It is named entity recognizer for Czech, so
exactly what we need. Problem is, that it is not finished yet and in C++
language, so there must be additional layer between Java and C++.
We researched possibilities about entity-object matching. Program will get
entity and assign it possible candidates from objects. This part could not be
purely automatic because of ambiguity, but if there will be high probability,
program should match object entity pair automatically.

Possible solutions for this entity-object matching are machine learning  methods
ranking or classification.

We looked for suitable Java library which provides machine learning techniques.
Candidates were JavaML and Weka. After testing these two libraries, we chose
JavaML, because it provides more machine learnig algorithms (some of them uses
Weka library). Weka's problem is that it was created for teaching purposes so
it's performance is not as good as JavaML.

There were two possible server architectures: standalone or embedded webserver
(Tomcat or Jetty). We considered usage of some technologies like Spring, CXF and
Hibernate for database layer.

\subsection{December}
After choosing technologies, we made some prototypes of each component. We
discussed a lot APIs (we need them for testing). For testing purposes, we must
provide some mock objects, so everyone implemented mock of his component.
NameTag was released, so we could start testing it and integrated it to our
project.

\subsection{January}
\subsection{February}
\subsection{March}
\subsection{April}
\subsection{May}
\subsection{June}
\subsection{July}
\subsection{August}