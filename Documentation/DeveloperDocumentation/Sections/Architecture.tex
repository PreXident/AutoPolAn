%This section cotains the system architecture
%%%%%%%%%%%%%%%%%%%%%%%%%%%%%%%%%%%%%%%%%%%%%%%%%%%%%%%%%%%%%%%%%%%%%%%%%%%%%%% (80 char rule :D)

%For every one: use as many diagrams as possible :) (author Petr)

The architecture of \textan\ is based on client-server model. Two main components
are the \textan\ server and \textan\ client which communicate via W3C web services
(SOAP protokol).

\subsection{Server architecture}

The server architecture is based on a logical multilayered architecture. It
consists from a presentation layer, a service layer, a business logic layer, 
a persistence layer and a data layer.

%TODO where to place commands in architecture?? (author Petr)

%TODO server achitecture image: assignee: Petr (author Petr)

%TODO description of layers
% assignee: Petr, Venca (persistence layer) (author Petr)

\subsubsection{Named entity recognizer architecture}

%TODO here should be description of nametag integration (recognizing, leraning etc.), nametag..
% assignee: Jakub (author Petr) 

\subsubsection{Object assigner architecture} %or simply TextPro?

%TODO here should be description of TextPro 
% assignee: Tam (author Petr)

\subsection{Client architecture}

%TODO here should be description of client architecture (MVC etc.)
% assignee: Adam (author Petr)
