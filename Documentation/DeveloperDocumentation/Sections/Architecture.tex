%This section cotains the system architecture
%%%%%%%%%%%%%%%%%%%%%%%%%%%%%%%%%%%%%%%%%%%%%%%%%%%%%%%%%%%%%%%%%%%%%%%%%%%%%%% (80 char rule :D)

%For every one: use as many diagrams as possible :) (author Petr)

The architecture of \textan\ is based on client-server model. Two main components
are the \textan\ server and \textan\ client which communicate via W3C web services
(SOAP protokol).

\subsection{Server architecture}

The server architecture is based on a logical multilayered architecture. It
consists from a presentation layer, a service layer, a business logic layer, 
a persistence layer and a data layer.

%TODO where to place commands in architecture?? (author Petr)

%TODO server achitecture image: assignee: Petr (author Petr)

%TODO description of layers
% assignee: Petr, Venca (persistence layer) (author Petr)

\subsubsection{Named entity recognizer architecture}

%TODO here should be description of nametag integration (recognizing, leraning etc.), nametag..
% assignee: Jakub (author Petr) 

\subsubsection{Object assigner architecture} %or simply TextPro?

%TODO here should be description of TextPro 
% assignee: Tam (author Petr)

\subsection{Client architecture}

%TODO here should be description of client architecture (MVC etc.)
% assignee: Adam (author Petr)

\textan\ Client is a JavaFX application which naturaly forces Model-View-Controller
pattern.

Model classes can be found in package cz.cuni.mff.ufal.textan.core. It contains
mainly client side representation of object, entity, relation etc. This way
clients do not have to use directly websevice representation, so when it is
altered, changes needed in client side are localized into this package. Moreover
there is Client class which completely hides the webservices and transformation
webservice data representation to client side representation.

Subpackages of the Core package focus on individual client side tasks, like
displaying graphs and processing report in a pipeline. The pipeline/wizard
approach was chosen because because amount of choices and required tasks would
be overhelming to users if they are all provided at once.

View descriptions are located in cz.cuni.mff.ufal.textan.gui in resources
directory, mainly names *.fxml and *.css. They also uses several custom and
modified components located in main cz.cuni.mff.ufal.textan.gui package along
with controller classes.

Controller classes are concentrated into cz.cuni.mff.ufal.textan.gui package,
mainly named *Controller. There are also some custom and modified view classes
used by view descriptions.