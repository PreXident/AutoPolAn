% Chapter: Requirements spec.

\section{Functional requirements}
\subsection{Mandatory functional requirements}
This section contains a list of functional must be implemented in the final solution.

\begin{itemize}
	\item The system recognizes entities in a document and determines their type.
	\item The system assigns recognized entities to objects from the database.
	\item An user will be able to edit recognized entities (add an entity, remove
	an entity, change a type of of an entity, change a range of an entity).
	\item An user will be able to edit assignments of objects to entities.
	\item An user will be able to mark relations between objects.
	\item An user will be able to add types of entities (respectively objects)
	and relations.
	\item The system will continuously learn to recognize entities and to assign
	objects to entities from processed documents confirmed by an user.
	\item An user will be able to merge objects saved in the database.
	\item An user will be able to browse objects and relations between them stored
	in the database as lists and as graphs.
	\item An user will be able to use to browse the database these kinds of search:
	\begin{itemize}
		\item a full text search in documents,
		\item a searching of objects or relations by types,
		\item a searching of objects by their aliases,
		\item and combinations of these methods.
	\end{itemize}
	\item An user will be able to suspend work on a currently processed document.
	\item Ongoing changes will be stored locally, they will be stored in the database
	collectively after completion work on the document.
	\item The database must log (revise) all performed changes.
\end{itemize}

\subsection{Optional functional requirements}
\comment{Petr}{Add some explanation. Or not?}
\begin{itemize}
	\item An user will be able to cancel a merging of objects followed by
	a semi-automatic assignment of object occurrences, that was recognized while
	they were merged, to divided objects.
	\item The system will recognize relations between objects in a document by
	relations that an user marked before in other documents.
	\begin{itemize}
		\item An user will be able to edit recognized relations between objects
		in a document.
	\end{itemize}
	\item An user will be able to use to browse the database more complex kinds
	of search:
	\begin{itemize}
		\item a search of connection between objects,
		\item a search of relations by anchors in documents.
	\end{itemize}
\end{itemize}

\section{Non-functional requirements}
\begin{itemize}
	\item The system must support a parallel connection of more users.
	\item The system must be able to recover from a crash of server or client.
	\item The system must be connectible with other systems.
	\item The system must reasonably handle a unavailability of server.
	\item An updating of auto-recognition, a learning from new examples,  may take
	a longer time, the system should use an old model during the learning.
	\item The system must be usable without an Internet connection, e.g. in
	a closed intranet.
\end{itemize}

\section{Fixed limits} \comment{Petr}{Is it correct name for this section?}
\begin{itemize}
	\item Entities are contiguous and not overlap in a document.
	\item An user won't be able to edit a data stored in a database. The only form
	of update will be insertion of a new document and its annotations.
	\item A parallel annotation of a document won't be supported.
	\item The system won't solve an authentization and  an authorization of users.
	\item The system won't solve a backup.
\end{itemize}


