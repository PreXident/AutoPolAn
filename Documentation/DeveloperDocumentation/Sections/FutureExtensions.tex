% Chapter: POSSIBILITIES OF FUTURE EXTENSIONS

\section{Relation Recognition}
The automatic recognition of relations between objects is the most missing thing
in the current version of \textan{}. However, because this task has been taken
into account from the very beginning of the development and because it is one of
the optional requirements, \textan{} contains the whole infrastructure except
the core component itself.

The recognition can be based on many different criteria, same as relations.
For example: the presence of objects in the same document, the kind of verbs,
which are the anchors for relations in the most cases, their valency and
relations which users marked before in other documents. The component could be
probably based on some machine learning method and above mentioned attributes.

\section{Searching}
Although \textan{} provides several search methods, they are weak compared to
possibilities that graphs offer. Consider for example an environment of police
reports and a task in which you need to find all persons who know a killer. The
graph contains this information, if good model is defined for the domain, but
how can you ask for them?

It implies at least usage of some graph query language and creation of some
graph indexing over the data. Or in better case usage of different data storage. 

\section{Underlaying Database}
In the case that graph operations become more important, it will be better to use
a different data store, for example a NoSql database based on RDF or a graph
database. There are many implementations of these databases, both commercial and
open-source, e.g. OpenLink Virtuoso\footnote{\url{http://virtuoso.openlinksw.com/}},
or Neo4j\footnote{\url{http://www.neo4j.org/}}.

\section{Restrictions for Objects in Relation}
As it is noted in user documentation section \ref{USR-sec:Domain},
\textan{} is very universal and needs some conventions to provide good results,
because there is no way how to enforce the proper use. The system cannot work
properly without conventions anyway, but some extension enforcing restrictions
can improve overall outcomes.

The most natural improvement is restricting objects in relations. Consider for
example relation type "murder". There must probably be at least one object of
type "person" with role "victim" in the relation of this type, also there
should be an object of type "person" with role "murderer", an object of type
"address" with role "crime place" and so on.

\section{Client}
The \textan{} client could be improved in many ways as any graphic application.
Some examples:

\begin{itemize}
	\item Filtering in documents could display exact location of matched text
	\item There could be a mechanism to localize object and relation types and
	relation roles
	\item Inner windows could use some layout framework
	\item Inner windows could transform to Outer stages and vice-versa on the
	run
\end{itemize}
