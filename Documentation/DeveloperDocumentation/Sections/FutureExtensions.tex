% Chapter: POSSIBILITIES OF FUTURE EXTENSIONS

The chapter introduces some areas, that should be in our opinion extended or
improved. The list does not contain all possible extensions, of course.

\section{Relation Recognition}
The automatic recognition of relations between objects is the most missing
feature in the current version of \textan{}. However, because this task has been
taken into account from the very beginning of the development and because it is
one of the optional requirements, \textan{} contains the whole infrastructure
except for the core component itself.

The recognition can be based on many different criteria, same as relations.
For example: the presence of objects in the same document, the kind of verbs,
which are the anchors for relations in the most cases, their valency and
relations which users marked before in other documents. The component could be
probably based on some machine learning method and above mentioned attributes.

\section{Searching}
Although \textan{} provides several search methods, they are weak compared to
possibilities that graphs offer. Consider for example an environment of police
reports and a task in which you need to find all persons who know a killer. The
graph contains this information, if good model is defined for the domain, but
how can you ask for them?

It implies at least usage of some graph query language and creation of some
graph indexing over the data. Or in better case usage of different data storage. 

\section{Underlying Database}
In the case that graph operations become more important, it will be better to
use a different data store, for example a NoSql database based on RDF or a graph
database. There are many implementations of these databases, both commercial and
open-source, e.g. OpenLink Virtuoso\footnote{\url{http://virtuoso.openlinksw.com/}},
or Neo4j\footnote{\url{http://www.neo4j.org/}}.

\section{Restrictions for Objects in Relation}
As it is noted in user documentation Section \ref{USR-sec:Domain},
\textan{} is very universal and needs some conventions to provide good results,
because there is no way how to enforce the proper use. The system cannot work
properly without conventions anyway, but some extension enforcing restrictions
can improve overall outcomes.

The most natural improvement is restricting objects in relations. Consider for
example relation type ``murder''. There must probably be at least one object of
type ``person'' with role ``victim'' in the relation of this type, also there
should be an object of type ``person'' with role ``murderer'', an object of type
``address'' with role ``crime place'' and so on.

\section{Multiple Languages at Once}
Although \textan{} can be configured for multiple languages, mainly for Czech
and English (see Section \ref{USR-sec:Lang} in the user documentation), it
support only one language at once. It can be useful to support multiple
languages at once. For example, when \textan{} is used by police for analyze
their reports, it can be helpful to manage reports from foreign colleagues in
same system. A similar example is any multinational company and its documents.

This change affects multiple areas in \textan{}, especially recognizers and
the full text indexing and searching, because they use different procedures for
each language and it is difficult to find any universal solution for them. It
means that the system must know which language is currently used and must choose
the right method for processing. It can be solved by multiple models for
recognizers, one for each language, and multiple analyzers for the searching and
the language will have to be defined explicitly.

\section{Better Czech Full Text Search}

Although Apache Lucene that is used for the text indexing in \textan is very
powerful tool, it is results are highly depend on a natural language analysis.
The Lucene contains an Analyzer (see Section \ref{USR-sec:Lang} in the user documentation)
for Czech language, but it isn't very good. We tried to implement a better analyzer
based lemmatization via the MorphoDiTa tool (see Section \ref{sec:UsedTechnologies}),
but it uses different approach then the Lucene, so it remained unfinished.

\section{Client Improvements}
The \textan{} client could be improved in many ways as any graphic application.
This section contains a few examples that could improve user experience and
efficiency.

When users filter documents by full text, only the list of documents matching
the criteria is displayed. No information about the position of the match etc.
is provided. Lucene offers such functionality, so if webservice interface is
extended, the client could be altered to display exact location of the matched
text.

Although full localization of the application is implemented, the object and
relation type names are directly fetched from the database. This could be a
major issue if feature ``multiple languages at once'' is implemented as it would
be required to use names understood by all users. Hence the mechanism for
localization of object and relation types could really improved the chances of
success in international multilingual environment.

The inner windows are only simple containers and do not have advanced layout
possibilities that many operating system provides and users are used to.
Therefore if more sophisticated layout framework is developed or external
component proving such features is employed, the user experience and
satisfaction could improve a lot.

Despite the fact that users can choose whether they want to use inner windows
or outer stages, there is no way to transform between them on the run. This is
a feature that should not be hard to implement, but it could be used by some
users.
