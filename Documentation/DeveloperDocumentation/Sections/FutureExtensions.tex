% Chapter: POSSIBILITIES OF FUTURE EXTENSIONS

\section{Relation recognition}
The automatic recognition of relations and objects in them is most missing thing
in the current version of \textan{}. Because it was calculated with this task from
the beginning of development of the system and because it is one of optional
requirements, \textan{} contains whole infrastructure except the core component
itself.

The recognition can be based on many different things, same as relations.
For example: a presence of objects in same document, a kind of verbs, which will
be the anchor for relations in documents in the most cases, and their valency and
relations which an user marked before in other documents. The component will
be probably based on some machine learning method and above mentioned attributes.

\section{Searching}
Although \textan{} provides several search methods, they are weak compared to
possibilities of graphs. Consider for example a environment of a police reports
and a task in which you need to find all persons who know a killer. The graph
contains these information, if there is defined a good model of the environment,
but how can you ask for them?

It implies at least a usage of some graph query language and a creation of some
graph indexing over the data or a usage of different data storage in a better case. 

\section{Underlaying database}
In the case that graph operations become more important, it will be better to use
a different data store, for example a NoSql database based on RDF or a graph
database. There are many implementations of these databases, both commercial and
open-source, e.g. OpenLink Virtuoso\footnote{\url{http://virtuoso.openlinksw.com/}},
or Neo4j\footnote{\url{http://www.neo4j.org/}}.

\section{Restrictions for objects in relation}
As it is noted in user documentation \comment{Petr}{Add reference to user doc.},
\textan{} is very universal and needs some conventions to serve good results,
because there is no way how to enforce the proper use. The system cannot work
properly without conventions anyway, but some software enforceable restrictions
can improve a usage of it.

The most natural is to restrict objects in relations. Consider for example a relation
type "murder". There must probably be at least one object of a type "person" with
a role "victim" in the relation of this type, also there should be a object of
a type "person" with a role "murderer", a object of  a type "address" with a role
"crime place" and so on.
