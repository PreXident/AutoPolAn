%%%%%%%%%%%%%%%%%%%%%%%%%%%%%%%%%%%%%%%%%%%%%%%%%%%%%%%%%%%%%%%%%%%%%%%%%%%%%%%

\subsection{Used Technologies}

This section lists all technologies and libraries used in \textan.

%TODO are links to the libraries missing intentionally? Adam
\paragraph{Apache CXF}
Apache CXF is an open-source Web services framework. CXF implements the JAX-WS 
standard APIs, is configurable through Spring and can be embedded into Jetty,
Tomcat or other servlet containers.

\paragraph{ControlsFX}
ControlsFX is a Java8 open source library aiming at enhancing the programming
experience with standard JavaFX. It provides wide selection of new and improved
components to make everyday use of JavaFX even easier. \textan\ uses it mainly for
its high quality and user-friendly standard dialogs which are incomprehensibly
and sorely missing in JavaFX.

\paragraph{Hibernate ORM}
%TODO Venca
Hibernate ORM (Hibernate in short) is an object-relational mapping library for
Java language, providing a framework for mapping an object-oriented domain
model to a traditional relational database. Hibernate solves object-relational
impedance mismatch problems by replacing direct persistence-related database
accesses with high-level object handling functions.

\paragraph{Hibernate Search}
Hibernate Search is a library (or Hibernate ORM extension), which integrates
the full text library functionality from Apache Lucene in the Hibernate and
JPA model. It offers full-text search support for objects stored by Hibernate
ORM, Infinispan and other sources. Think of it as Google\texttrademark\ for
entities: search words with text, order results by relevance and find by
approximation (fuzzy search).

\paragraph{Java-ML}
%TODO Kuba/Tam

%ML (author Tam)
First of all, let's talk Machine Learning (ML), a hot and developing subfield of computer
science in general, of computational linguistics specifically. The key of ML is 
that it can learn from experience to improve the performance with regards to a 
specific task. When the amount of experience increase (collecting more data, human
intervention, etc), the performance is like to go up as well. 

%JAVA-ML (author Tam)
Java-ML is a machine learning library for Java. It is a collection of machine
learning and data mining algorithms. It consists of a few Machine learning 
techniques such as data pre-processing, clustering, classification, feature 
selection, regression, ensemble learning, voting. 

\paragraph{Jetty}
Jetty is a pure Java-based HTTP (Web) server and Java Servlet container. It can
be embedded into other applications through Jetty API.

\paragraph{JUNG}
JUNG (Java Universal Network/Graph Framework) is older Java library for
displaying graphs, still widely used in Java world, well documented and easily
extensible. Sadly, it does not support JavaFX, but only older Java GUI framework
SWING. Fortunately, thanks to new JavaFX8 SwingNode enabling usage of SWING
components in JavaFX scenes it was possible to seaminglessly integrate JUNG into
JavaFX \textan\ Client.

\paragraph{JFXtras}
JFXtras is another JavaFX8 enhancing project. \textan\ Client uses it mostly for
its window management capabilities enabling extensible use of inner windows
embedded into the main window. Another used component is BigDecimalField which
brings common Spinner component into JavaFX.

\paragraph{MorphoDiTa}
Morphological Dictionary and Tagger is an open-source tool for morphological
analysis of natural language texts. It performs morphological analysis, 
morphological generation, tagging and tokenization and is distributed as
a standalone tool or a library, along with trained linguistic models. In
the Czech language, MorphoDiTa achieves state-of-the-art results with 
a throughput around 10-200K words per second. MorphoDiTa is a free software
under LGPL license and the linguistic models are free for non-commercial use
and distributed under CC BY-NC-SA license, although for some models the original
data used to create the model may impose additional licensing conditions.

\paragraph{NameTag}
NameTag is an open-source tool for named entity recognition (NER). NameTag 
identifies proper names in text and classifies them into predefined categories,
such as names of persons, locations, organizations, etc. NameTag is distributed
as a standalone tool or a library, along with trained linguistic models.
In the Czech language, NameTag achieves state-of-the-art performance
(Straková et al. 2013). NameTag is a free software under LGPL license and the 
linguistic models are free for non-commercial use and distributed under CC 
BY-NC-SA license, although for some models the original data used to create
the model may impose additional licensing conditions.

\paragraph{PretopoLib}
Although JUNG is mature and well thought library, it lacks one important feature
out-of-the-box: displaying hypergraphs. For this reason JUNG graph rendering
part of PretopoLib is used. PretopoLib is a library mainly focusing on
pretopology and graph displaying is rather byproduct. However it fully meets our
requirements and its author kindly provided us the sources so we could fix one
inconvenient detail.

\paragraph{SLF4J \& Logback}
Simple Logging Facade for Java (SLF4J) provides a Java logging API by means
of a simple facade pattern. The underlying logging backend is determined
at runtime by adding the desired binding to the classpath and may be
java.util.logging, log4j or logback.

\paragraph{Spring Framework}
The Spring Framework is an open source application framework and inversion
of control container for the Java platform. The framework's core features
can be used by any Java application, but there are extensions for building
web applications on top of the Java EE platform.
