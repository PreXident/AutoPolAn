%%%%%%%%%%%%%%%%%%%%%%%%%%%%%%%%%%%%%%%%%%%%%%%%%%%%%%%%%%%%%%%%%%%%%%%%%%%%%%%

%\subsection{Used Technologies}
\label{sec:UsedTechnologies}

This section lists all technologies and libraries used in \textan{} and
reasoning behind choosing them.

\paragraph{Spring Framework\footnote{\url{http://projects.spring.io/spring-framework/}}}
The Spring Framework is an open source application framework and inversion
of control container for the Java platform. The framework's core features
can be used by any Java application, but there are extensions for building
web applications on top of the Java EE platform. Apart from Dependency injection
we use also Spring AOP for transaction handling.

\paragraph{Apache CXF\footnote{\url{http://cxf.apache.org/}}}
Apache CXF is an open-source Web services framework. CXF implements the JAX-WS
standard APIs, is configurable through Spring and can be embedded into Jetty,
Tomcat or other servlet containers.

\paragraph{ControlsFX\footnote{\url{http://fxexperience.com/controlsfx/}}}
ControlsFX is a Java8 open source library aiming at enhancing the programming
experience with standard JavaFX. It provides wide selection of new and improved
components to make everyday use of JavaFX even easier. The \textan{} client uses
it mainly for its high quality and user-friendly standard dialogs which are
incomprehensibly and sorely missing in JavaFX.

\paragraph{Hibernate ORM\footnote{\url{http://hibernate.org/orm/}}}
Hibernate ORM (Hibernate in short) is an object-relational mapping library for
Java language, providing a framework for mapping an object-oriented domain
model to a traditional relational database. Hibernate solves object-relational
im\-pedance mismatch problems by replacing direct persistence-related database
accesses with high-level object handling functions. It was chosen because of
possible extension of full text search independent of the underlying database.

\paragraph{Hibernate Search\footnote{\url{http://hibernate.org/search/}}}
Hibernate Search is a library (or Hibernate ORM extension), which integrates
the full text library functionality from Apache Lucene in the Hibernate and
JPA model. It offers full-text search support for objects stored by Hibernate
ORM, Infinispan and other sources. Think of it as Google\texttrademark{} for
entities: search words with text, order results by relevance and find by
approximation (fuzzy search).

\paragraph{JCommander\footnote{\url{http://jcommander.org/}}}
JCommander is a very small Java framework that makes it trivial to parse command
line parameters. It was chosen because of easy-to-use annotation API and no
external dependencies.

\paragraph{Jetty\footnote{\url{http://www.eclipse.org/jetty/}}}
Jetty is a pure Java-based HTTP (Web) server and Java Servlet container. It can
be embedded into other applications through Jetty API. It is simple,
lightweight, wide spread and fast when embedded.

\paragraph{JUNG\footnote{\url{http://jung.sourceforge.net/}}}
JUNG (Java Universal Network/Graph Framework) is an older Java library for
displaying graphs for Swing graphical framework, still widely used in Java
world, well documented and easily extensible, which are main reason why it was
chosen. Sadly, it does not support JavaFX. Fortunately, thanks to new JavaFX8
SwingNode enabling usage of SWING components in JavaFX scenes it was possible to
seamlessly integrate JUNG into JavaFX \textan{} Client.

\paragraph{JFXtras\footnote{\url{http://jfxtras.org/}}}
JFXtras is another JavaFX8 enhancing project providing components that
most Java developers need in daily work, but they are currently missing
from JavaFX. \textan{} Client uses it mostly for its window management
capabilities enabling extensible use of inner windows embedded into the main
window. Also several other useful components are used.

\paragraph{MorphoDiTa\footnote{\url{http://ufal.mff.cuni.cz/morphodita}}}
Morphological Dictionary and Tagger is an open-source tool for morphological
analysis of natural language texts. It performs morphological analysis, 
morphological generation, tagging and tokenization and is distributed as
a standalone tool or a library, along with trained linguistic models. In
the Czech language, MorphoDiTa achieves state-of-the-art results with 
a throughput around 10-200K words per second. MorphoDiTa is a free software
under LGPL license and the linguistic models are free for non-commercial use
and distributed under CC BY-NC-SA license, although for some models the original
data used to create the model may impose additional licensing conditions.

\paragraph{NameTag\footnote{\url{http://ufal.mff.cuni.cz/nametag}}}
NameTag is an open-source tool for named entity recognition (NER). NameTag 
identifies proper names in text and classifies them into predefined categories,
such as names of persons, locations, organizations, etc. NameTag is distributed
as a standalone tool or a library, along with trained linguistic models.
In the Czech language, NameTag achieves state-of-the-art performance
(Straková et al. 2013). NameTag is a free software under LGPL license and the 
linguistic models are free for non-commercial use and distributed under CC 
BY-NC-SA license, although for some models the original data used to create
the model may impose additional licensing conditions. There are more similar
tools, but we have chosen this because of java bindings, modern design and
support of more languages.

\paragraph{PretopoLib\footnote{\url{http://pretopolib.complexica.net/}}}
PretopoLib is a Java library based on pretopology which is mathematical theory
for analysis, modeling and simulation in several domains: human and social
sciences, game theory, graph theory extension, networks and discrete spaces.
Although JUNG is mature and well thought library, it lacks one important feature
out-of-the-box: displaying hypergraphs. PretopoLib offers such functionality
as a byproduct. However it is eye pleasing and fully meets our requirements and
its author kindly provided us the sources so we could fix one inconvenient
detail.

\paragraph{SLF4J \& Logback\footnote{\url{http://www.slf4j.org/}, \url{http://logback.qos.ch/}}}
Simple Logging Facade for Java (SLF4J) provides a Java logging API by means
of a simple facade pattern. The underlying logging backend is determined
at runtime by adding the desired binding to the classpath and may be
java.util.logging, log4j or logback.

\paragraph{WEKA\footnote{\url{http://www.cs.waikato.ac.nz/ml/weka/}}}
Weka is a collection of machine learning algorithms for data mining tasks.
The algorithms can either be applied directly to a dataset or called from your
own Java code. Weka contains tools for data pre-processing, classification,
regression, clustering, association rules, and visualization. It is also
well-suited for developing new machine learning schemes.

