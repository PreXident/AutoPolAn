Status: A dummy documentation - Text Analyser

\comment{Ondrej}{Insert links to project on github and testing server somewhere
into introduction}

\comment{Adam}{add reference to Appendix \ref{app:DVD}}

\paragraph{Motivation} In today's world, there is a large amount of unstructured written sources that contain structured information. An example are police reports. They contain information entities (names, addresses, phone numbers, weapons, drugs, etc..) and the relationships between them. This information must be transformed into representation which could be easily read by computers, for example, is written in the database and assign it to an already existing objects if they have the same meaning. This process is currently made manually, which is very time consuming. Similarly, same time-consuming processing documents can be found in other institutions or companies. Such problem of extracting structure information out of general documents has recently gained attention from both research and industry.\\
\\
This is the topic of our software project, called Text Analyser (TextAn).
TextAn is a client-server application for text analysis.
It is devoted to build a pleasant and effective tool for extracting structured data in Czech police reports. 
Our goal is to provide an effective extraction of structured data with user involvement.
It consists of three main contributions.
First, an efficient design of database for structured data is implemented.
Second, we aim to maximize the accuracy and performance of automatic detection of entities.
Third, TextAn is designed to provide a friendly method for users to confirm or adjust the system's suggestions.

Furthermore, the application allows users to add/remove/adjust the structured information.
They can change the relations, correct mistakes or provide further information.
The structured information could be displayed in graph-view,
\comment[Tam]{Adam, Ondrej}{profile? What is this about?}
which is a fancy way that users could look at his/her profile. 

\comment[anyone]{Adam}{add footnote linking chapter with used libraries when ready}
TextAn is implemented in Java 1.8 using Spring Framework, Hibernate database layer and other libraries.
The data are can be stored within any SQL-like relation database, e.g. MySQL.

The developer team consists of 5 students:

\begin{itemize}
\itemsep0em
\item Petr Fanta - server, interconnection of components
\item Duc Tam Hoang - entity-object matcher 
\item Adam Huječek - client
\item Václav Pernička - databse
\item Jakub Vlček - NameTag and MorphoDiTa integration
\end{itemize}
