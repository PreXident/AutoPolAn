
In today's world, there is a large amount of unstructured written sources that contain structured information. An example is police reports. They contain
information entities (names, addresses, phone numbers, weapons, drugs, etc..)
and the relationships between them. This information must be transformed into
representation which can be easily read by computers. For example, it can be
written into the database and assigned to already existing records stored in
the system. This process is currently made manually, which is very time 
consuming. Similarly, the same time-consuming processing documents can be found
in other institutions or companies. Such problem of extracting structure
information out of general documents has recently gained attention from both
research and industry.

The topic of our software project called Text Analyser (\textan{}) is to
deliver a pleasant and effective tool for extracting structured data from texts.
In many areas the sensitivity of the processed information does not allow fully
automatic extraction. Therefore the human element is involved into the process
to correct and adjust automatic results. Moreover these interventions are
exploited with machine learning techniques, so the results improve with time.

Although the first impulse came from the Czech Police needs, \textan{} is made
as general as possible to allow easy configuration for any other domain as
needed. It also is not language dependant, so it can process texts in completely
different languages if corresponding components are provided.

\comment{Petr}{Following two paragraphs needs rework - remove concrete libraries,
add platform independence, remove mention of schema}

\textan{} is a client-server application for text analysis. It provides
webservice interfaces, so it is easy to implement own client implementation to
meet specific user needs or to integrate it in a legacy system. The project
consists of three main contributions. Firstly, the database schema suitable for
given tasks is implemented. Secondly, the server integrating named entity
recognition and object assigning provides interface to access the database and
these components. Thirdly, the client offers a friendly method for users to
confirm or adjust the system's suggestions and explore the data stored in the
database displaying them in graph views.

\textan{} is implemented in Java 1.8 using Spring Framework, Hibernate database
layer and other libraries. For more information see Section
\ref{sec:UsedTechnologies}. The data can be stored within any SQL-like relation
database, e.g. MySQL.
