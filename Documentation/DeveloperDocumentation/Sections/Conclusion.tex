\comment{Ondrej}{Half page with summary: what textan is, what is described in
documentation, some of the most important extensions}

\textan{} offers semi-automatic machine reading \comment{Adam}{Can this be cited?}
text recognition. It extracts entities from documents and assigns them to objects
stored in the database. It utilizes existing solutions \emph{NameTag} and
\emph{MorphoDiTa} to recognize named entities and our own
implementation of object to entity assignment, both powered by machine learning.
Its functionality is easily accessible through SOAP based webservices described by
WSDL, either through the provided client application or through a custom
implementation offering additional features or integration to a legacy system.

This documentation presents both functional and non-functional formal requirements
and architecture designed to meet these requirements. Many third party libraries
are utilized to make sure the best possible results are achieved. The developer
documentation also lists the most important decisions made during the
development affecting various parts of the project and overall overview of the
development progress. The section Possible Future Extensions (see Section
\ref{sec:FutureExtensions}) contains several major enhancements that could improve
either user experience with the application or overall outcome of the system. For
example, the support of multiple languages at once would make \textan{} suitable
for usage in multilingual companies or international organizations that have to
process documents in many different languages. The most important possible
extension is however the recognition of relations between objects that could
eventually lead to fully automated processing of report without any kind of human
intervention. The work on automatic relation detection has actually already started as the master thesis of a colleague Richard Ejem~\cite{ejem14}

As of now, \textan{} satisfies all obligatory requirements from the official
assignment. Although not all optional requirements have been fulfilled, \textan{}
resulted in a coherent solution meeting the goals stated in the specification.
