% Section: Decisions

The section describes main decisions that was made during the development. 

\subsection{Fully-automatic or Semi-automatic processing}

Although automatic recognition can achieve very good results, there are many
sensitive areas where it is necessary to eliminate all errors, for example
police reports. For this reason we chose semi-automatic recognition, so users
can correct all mistakes before the system stores data into the database. 
Also if fully-automatic recognition is required, it is easy
to skip the human intervention entirely at cost of little overhead caused by
multiple calls of webservice operation because of pipeline decision described in
Section \ref{ssec:ReportPipeline}. Or a facade operation can be implemented to
mitigate this overhead.

Typical issues for a recognition are an ambiguity of words or names and a large
granularity of types. An example of ambiguity can be word Washington, it can mean
the state in USA, the city in USA or person with this surname. The second issue,
the large granularity of types, can be shown similarly. Consider object types "doctor",
"programmer" and "gardener" and a sentence with name of person. These problems are
highly dependent on a training data, a target domain and documents from users.

\subsection{Re-learning}

\subsection{System Decomposition}

\comment{Petr}{Describe why recognition is on the server, why web server is embedded etc.}

\subsection{Control of Processing by User}
\label{ssec:ReportPipeline}

\comment{Petr}{Pipeline? ... Or something about whole UI?}

\subsection{Stateless or Stateful Communication}


\comment{Petr}{At the moment, I don't know what should be here, but I think it is important. :D
Maybe something about improving models?}

\subsection{Database}

\comment{Petr}{There must be some thing about database}

First problem we have to face was choosing the type of database. As you may know there are other databases then relational, called \emph{NoSQL} databases, meaning \emph{Not Only SQL}. There are a few types of them, but the type that fits the most to our problem is graph database. The standard (relational) database is based on tables and relations between them. When you create a database like this you have to know the exact schema and any further changes in schema can be very hard to perform in already running system. On the other hand, relational databases are more type safe. You just can have constraints on columns that secure you meaningful information in the records. Graph databases basically look like a graph - there are nodes and edges, both can have any additional information any kind. These are suitable especially for graph queries that can't be done easily in relational database.

Anyway we decided to use the relational database for several reasons.
The graph databases are pretty new technology, it is not so time proven and we were too afraid of bugs possibilities. Apart from that relational database are well documented, all main bugs have been solved years ago and everyone knows what to expect and what is impossible.
We also wanted to use database for machine learning, which uses sequential access to the database. That is, as we know, faster in relational database, where can we only iterate through all records in a single table.

Although we have used relational database, this decision can be wrong and through the time it can come out that graph database can be much better solution. We expected that a little bit and used the DAO design pattern enabling to change the type of database without changing the application source code. This pattern is more described in section \ref{sec:PersistentLayer} about server architecture respectively persistent layer.

