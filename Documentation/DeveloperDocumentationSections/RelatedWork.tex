%This section provides details on other projects/publications which are in the similar fields.

Mining structured information from unstructured text documents has recently gained attention. Current approaches are still 
far from an fully automated and completely accurate processing. However, the challenge has led to a number of research and 
applications. Each works focused on solving a particular problem in the big pictures with regard to some specific languages. 

This section provides an insight into other contributions to the mining problem. Both research-oriented works and 
industry applications are taken into account.

% http://knowtator.sourceforge.net/
% http://knowtator.sourceforge.net/docs/Ogren_HLT-NAACL06_Demo_Abstract_Final.pdf
Knowtator is a general-purpose text annotation tool. Developed by scientists at Division of Biomedical Informatics, 
it uses Protégé knowledge-base as the database. As an early development of tool for annotating data, Knowtator has 
a number of limitations (OS dependent, no client-server, no automatic detection). The remarkable feature of Knowtator
is the ability to relate annotations to each other via the slot \textit{reference}.  The tool has been implemented as
a Protégé plug-in for wider-spread usage.


%http://www.aclweb.org/anthology/N/N13/N13-3004.pdf
%Anafora: A Web-based General Purpose Annotation Tool

Anafora is an open source web-based text annotation tool, developed by scientists of  University of Colorado at Boulder. It 
distinguishes itself from previous contribution in the field of OS platform. Before the introduction of Anafora, old tools 
were written as a local application in a local machine under the threat of data fragmentation. Anafora is a web-based tool 
with client-server structure. % OK, INTRODUCTION TO ANAFORA


Anafora is designed for the medical domain with a medical named entity tags, not a general domain. It is developed with a server in Django (a Python framework) and a client in jQuery (a JavaScript 
library). Besides, it stores annotations in each single XML files. By doing so, authors claim that the tool is agile and 
flexible. The project hierachy is designed to be the file/directory structures. The application provides an automatic detection
of named entities. Users could change/add details to the annotation by mouse or keyboard. Language choice is English.

The limitation of Anafora lies in the data structure. The complicated definitions like relations (or relation properties) 
among entities are not supported. It may be the trade-off when authors want to develop a light tools based on a simple
data structure such as XML.

% http://brat.nlplab.org/

BRAT rapid annotation tool is a web-based tool for text annotation, most useful to add note to existing text documents 
(taken from the introduction of BRAT). It is developed by Japanese. This tool does support the linking relation between 
two entities and even the link 
from entities to a definition outside documents (such as wikipedia).

The strong point of BRAT is the simplicity of usage. The tool offers a comprehensive visualization with mouse-based 
activities, a linking method to an external resources (such as Freebase, Wikipedia, and Open Biomedical Ontologies). A number 
of other useful functions are \textit{save, export from standoff format, search}. The brat standoff format is a simple 
format developed for the visualization of the tool, in which data is stored in two files: a text file \textit{.txt} and an
annotation file \textit{.ann}.

The weak point of the tool is that it does not support any automatic recognition of entities. All annotations have to be
done manually. This leads to a features that the tool could be applied on a wide range of languages. However, the 
limitation leads to a fact that BRAT is merely an editor for annotating.

