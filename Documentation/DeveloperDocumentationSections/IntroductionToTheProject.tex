Status: A dummy documentation - Text Analyser

Generally, Police often generate some documents with regards to a specific person (criminal records, conflict, ...). 
In the documents, there is a lot of information such as person name, date of birth, name of other persons, his/her 
location, other persons' location. They need an application to manage such documents in a convenient way and extract the 
important details without much human effort. This is the topics of our software project, call Text Analyser (TextAn).

TextAn is a client-server application for analysing text. It is to manage documents and exploit the information out of the 
documents. Given a collection of written texts (typically Police's reports) in Czech language, TextAn automatically breaks 
down the unstructured text to exploit the structured information. 

Furthermore, The application allows user to add/remove/adjust the structured information. They can change the relation, 
correct mistake or provide further information. The structured information could be seen in graph-view, which is a fancy 
way that user could look at his/her profile. 

Implementation of TextAn is recognised in Java 1.8 with Spring Framework. Information is stored with mysql database.

The developers's team consists of 5 students:

\begin{itemize}
\item Adam
\item Jakub
\item Peter
\item Tam
\item Vecca
\end{itemize}
