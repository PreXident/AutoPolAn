Status: A dummy documentation - Text Analyser

Generally, Police often generate some documents with regards to a specific person (criminal records, conflict, ...). 
In the documents, there is a lot of information such as person name, date of birth, name of other persons, his/her 
location, other persons' location. They need an application to manage such documents in a convenient way and extract the 
important details without much human effort. This is the topics of our software project, call Text Analyser (TextAn).

TextAn is a client-server application for analysing text. It is to manage documents and exploit the information out of the 
documents. Given a collection of written texts (typically Police's reports) in Czech language, TextAn automatically breaks 
down the unstructured text to exploit the structured information. 

Furthermore, The application allows user to add/remove/adjust the structured information. They can change the relation, 
correct mistake or provide further information. The structured information could be seen in graph-view, which is a fancy 
way that user could look at his/her profile. 

Implementation of TextAn is recognised in Java 1.8 with Spring Framework. Information is stored with mysql database.

The developers's team consists of 5 students:

\begin{itemize}
\item Adam
\item Jakub
\item Peter
\item Tam
\item Vecca
\end{itemize}

% TODO: PETER, please assemble two of my introductions 
Status: An even dummer introduction - Text Analyser

Textan (Text Analyser)

Generally, there is a huge amount of unstructured written texts which consist of structure information. It ranges from 
newspapers, medical bills to even personal letters. In every written documents, there are entities like names, address, 
dates, etc. These entities do not stand alone, there are relationship between them. The relationship could facilitate the 
problem of assigning an entity from new text to an existing objects. Such problem of extracting structure information out 
of a general documents has recently gained attention from both research and industry.

Out of all unstructured sources which contains structured information, Czech police reports is typical. They are the descriptions
with regard to a specific person, his/her profile and actions, criminal records and so on. Such information by no doubt is the most
valuable data in the whole documents. Currently, the task of extracting
it is done manually. In other words, there is a person who read through all documents, look at the database and match the
description in the document with existing data. This is not-surprisingly time-consuming and labour-intensive. 

Project Textan is devoted to build a pleasant and effective tool for extracting structured data in Czech police reports. 
Our goal is to provide an effective extraction of structured data with user involvement. It consists of three main contributions.
First, an efficient design of database for structured data is implemented. Second, we aim to maximize the performance of 
automatic detection of entities. Third, Textan is designed to provide a friendly method for users to confirm or adjust 
the system's suggestions. % This may be wrong. TODO: think how to make it clear
At first, the second task may achieve low accuracy, but the magic of machine learning could improve the performance through 
confirmation by users. 






















