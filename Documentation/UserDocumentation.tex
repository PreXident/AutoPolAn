\documentclass[12pt,a4paper]{report}

\usepackage[utf8]{inputenc}
\usepackage[english]{babel}

\usepackage{geometry}
\usepackage{xcolor}
\usepackage{amsmath}
\usepackage[some]{background}
\usepackage{listings}

%TODO select color coresponding to logo
\definecolor{titlepagecolor}{cmyk}{1,.60,0,.40}
\definecolor{javagreen}{rgb}{0.25,0.5,0.35}

\backgroundsetup{
scale=1,
angle=0,
opacity=1,
contents={
\begin{tikzpicture}[remember picture,overlay]
  \path [fill=titlepagecolor] (current page.west)rectangle (current page.north east); 
  \draw [color=white, very thick] (5,0)--(5,0.5\paperheight);
 \end{tikzpicture}}
}

\DeclareFixedFont{\bigsf}{T1}{phv}{b}{n}{1.5cm}

\usepackage[unicode,colorlinks=true]{hyperref}
\hypersetup{pdftitle=TextAn - user documentation}
\hypersetup{pdfauthor={Petr Fanta, Duc Tam Hoang, Adam Huječek, Václav Pernička, Jakub Vlček}}
\hypersetup{linkcolor=black, citecolor=black, urlcolor=black, filecolor=black}

\makeatletter
\def\@makechapterhead#1{
  {\parindent \z@ \raggedright \normalfont
   \Huge\bfseries \thechapter. #1
   \par\nobreak
   \vskip 20\p@
}}
\def\@makeschapterhead#1{
  {\parindent \z@ \raggedright \normalfont
   \Huge\bfseries #1
   \par\nobreak
   \vskip 20\p@
}}
\makeatother

\def\chapwithtoc#1{
\chapter*{#1}
\addcontentsline{toc}{chapter}{#1}
}

\setcounter{secnumdepth}{3}
\setcounter{tocdepth}{3}

\lstdefinelanguage{properties}{% new language for listings
  basicstyle=\ttfamily,
  sensitive=false,
  morecomment=[l]{\#},      % comment
  morestring=[b]",          % string def
  commentstyle=\color{javagreen},
  basicstyle=\small
}

\setkeys{Gin}{resolution=96}

\newcommand{\textan}{\emph{TextAn}}

\begin{document}

%TODO better titlepage
\begin{titlepage}
\BgThispage

\newgeometry{left=2cm,right=6cm,bottom=2cm}
\vspace*{0.3\textheight}
\noindent
\textcolor{white}{\bigsf TextAn}

\vspace*{1cm}
\noindent
\textcolor{white}{\Huge\textbf{\textsf{User Documentation}}}

\vspace*{2cm}\par
\noindent
\begin{minipage}{0.35\linewidth}
\textbf{Authors} \\
Petr Fanta \\
Duc Tam Hoang \\
Adam Huječek \\
Václav Pernička \\
Jakub Vlček\vspace{40pt} \\
\textbf{Supervisor} \\
Ondřej Bojar\vspace{40pt} \\
\textbf{Version} \\
0.1\vspace{40pt} \\
\textbf{Date} \\
\today \\
\end{minipage}

%TODO add logo

\end{titlepage}
\restoregeometry

\pagenumbering{roman}
\tableofcontents

%\chapter*{Intro}
%\addcontentsline{toc}{chapter}{Intro}


\chapter{User Guide}
\pagenumbering{arabic}

\section{Introduction}

TODO what is textan

TODO basic usage

\subsection{System overview}


\subsection{Glossary}

TODO definitions of terms in documentation and applications

\subsection{Conventions} %is this section needed?



\chapter{Administrator guide}

%TODO client instalation guide in the administrator guide or user guide?

\section{Server}

\subsection{Installation guide}


This section will show you how to install and build \textan\ server. 
%An installation guide for a server side application of \textan.

\subsubsection{Prerequisites}
\label{sssec:SerInstPre}

The \textan\ server requires installation of \emph{Java 8 JRE \footnote{We recommend to use JRE from \emph{Oracle}, 
available at: \url{http://www.oracle.com/technetwork/java/javase/downloads/index.html}}} or a later version and a 
relational database system (e.g. \emph{MySQL}). The server should be platform independent, but it depends on native 
and 3rd party libraries. 
Supported operating systems in the distribution of the \textan\ server are Linux (32 bit and 64 bit) and 
Windows (32 bit and 64 bit). % I am thinking ...

\subsubsection{Installation}

It is best to unpack an archive of the \textan\ server distribution into a directory. 
Once it is unpacked, the directory should contain server binaries and scripts. % TAM: somehow vague meaning

\subsubsection{Basic configuration}

Before the first start, it is necessary to set up the server and the database.

TODO write how to setup server

TODO add some link to \ref{sec:ServerSettings}

\subsubsection{Starting the server}

The server can be started by starting scripts in root directory. 
Scripts for different operating systems unfortunately aren't equivalent. % TAM: somehow vague meaning. equivalent = similar??

The start script for the Linux-based operating system (\emph{run.sh}) runs the server application as daemon and can be used 
to run the server as system service.

The start script for the Microsoft Windows OS is less powerful, only runs the server application. 
To run the server as a background process, the Windows Service is required. 
Unfortunately, there are no components that ship with the Jetty Distribution to make it a formal Windows Service. % what does ship mean?
However, we recommend the use of \emph{Apache ProcRun's Daemon\footnote{More information can be found on 
\url{https://commons.apache.org/proper/commons-daemon/procrun.html} and in JavaDoc documentation for 
\emph{cz.cuni.mff.ufal.textan.server.AppEntry} class.}}.

\subsection{Settings}
\label{sec:ServerSettings}

\subsubsection{Web server settings}

%connector
\paragraph{server.connector.host} The particular interface to listen on. If not set or 0.0.0.0, the web server listens on port on all interfaces.

\paragraph{server.connector.port} The port to listen on. If not set, the web server listens on port 9500.

%thread pool
\paragraph{server.threadPool.maxThreads} The maximum number of threads in web server thread pool. It determines a maximum number of simultaneously opened connections. The default value is 200.

\paragraph{server.threadPool.minThreads} The minimum number of threads in web server thread pool. The default value is 8.

\paragraph{server.threadPool.idleTimeout} The time in milliseconds that the connection can be idle before it is closed.

\subsubsection{Database connection settings}

%jdbc
\paragraph{jdbc.driverClassName} 

\paragraph{jdbc.url} 

\paragraph{jdbc.user}

\paragraph{jdbc.pass}


%c3p0 (http://www.mchange.com/projects/c3p0/#configuration_properties)
\paragraph{c3p0.maxPoolSize}

\paragraph{c3p0.minPoolSize}

\paragraph{c3p0.initialPoolSize}

\paragraph{c3p0.acquireIncrement}

\paragraph{c3p0.maxIdleTime}

\paragraph{c3p0.checkoutTimeout}

\paragraph{c3p0.maxStatements}

\paragraph{c3p0.maxStatementsPerConnection}

\paragraph{c3p0.idleConnectionTestPeriod}

%hibernate
\paragraph{hibernate.dialect}

\paragraph{hibernate.show\_sql}

\subsubsection{Named entity recognizer settings}

\subsection{Data formats}

\section{Client}

\subsection{Installation guide}

\subsubsection{Prerequisites}

The \textan\ client requires installation of \emph{Java 8 JRE} or later version same as the server 
(see \ref{sssec:SerInstPre}).

\subsubsection{Installation}

Unpack an archive with the \textan\ client distribution into any directory. 
The archive contains a .jar file and starting scripts for the \emph{Microsoft Windows} and Linux-based
operating systems.

\subsubsection{Basic configuration}

If the client doesn't connect to the default server, it is necessary to configure a server location. 
The settings of server location are in \emph{TextAn.properties\footnote{A description of format of .properties file 
can be found at \url{http://en.wikipedia.org/wiki/.properties}}} in the directory with the client. 
Simply edit or add following lines to properties file:
\begin{lstlisting}[frame=single,language=properties]
#url of the document processor
url.document=http://localhost:9500/soap/document
#url of the document processor wsdl
url.document.wsdl=http://localhost:9500/soap/document?wsdl
#url of the data provider
url.data=http://localhost:9500/soap/data
#url of the data provider
url.data.wsdl=http://localhost:9500/soap/data?wsdl
\end{lstlisting}

\subsubsection{Starting the client}

The .jar file from the distribution archive is executable, but we recommend to use starting 
scripts\footnote{run.bat for the Microsoft Windows and run.sh for operating systems based on Linux}.

\subsection{Settings}

\end{document}
